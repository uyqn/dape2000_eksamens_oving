\documentclass[a4paper,norsk]{article}
\usepackage[utf8]{inputenc}
\usepackage[T1]{fontenc,url}
\usepackage{babel,textcomp}
\usepackage{amsmath}
\usepackage{polynom}
\usepackage[parfill]{parskip}
\usepackage{color}
\usepackage[dvipsnames]{xcolor}
\usepackage{pgfplots}
\urlstyle{sf}
\usepackage{venndiagram}
\usepackage[numbered,framed]{matlab-prettifier}
\usepackage{epstopdf}
\usepackage{listingsutf8}

\usepackage{natbib} % Required to change bibliography style to APA
\usepackage{graphicx} % Required for the inclusion of images
\usepackage{siunitx} % Provides the \SI{}{} and \si{} command for typesetting SI units

\usepackage{geometry} % to change the page dimensions
\geometry{a4paper} % or letterpaper (US) or a5paper or....

\usepackage{graphicx} % support the \includegraphics command and options

\usepackage{booktabs} % for much better looking tables
\usepackage{array} % for better arrays (eg matrices) in maths
\usepackage{paralist} % very flexible & customisable lists (eg. enumerate/itemize, etc.)
\usepackage{verbatim} % adds environment for commenting out blocks of text & for better verbatim
\usepackage{subfig} % make it possible to include more than one captioned figure/table in a single float
% These packages are all incorporated in the memoir class to one degree or another...

\usepackage{amsmath,amsfonts,amssymb,amscd,amsthm,xspace}% for mathematical symbols
\usepackage[colorlinks=true,linkcolor=black]{hyperref} % for hyperreferences with black color

\newcommand{\eksamen}[1]{\,
	\part*{Eksamen #1}
	\markright{}
	\addcontentsline{toc}{part}{Eksamen #1}
	\setcounter{subsection}{0}
}
\newcommand{\matte}{\,
	\section*{Matematikk}
	\markright{}
	\addcontentsline{toc}{section}{Matematikk}
}
\newcommand{\statistikk}{\,
	\section*{Statistikk}
	\markright{}
	\addcontentsline{toc}{section}{Statistikk}
}
\newcommand{\oppgave}{\,\subsection{}}
\newcommand{\deloppgave}{\, \subsubsection{}}

\newenvironment{utregning}{\begin{eqnarray*}}{\end{eqnarray*}}
\newenvironment{likning}{\begin{equation*}}{\end{equation*}}
\renewcommand{\labelenumi}{\roman{enumi})}

\newcommand{\dd}{\,\mathrm{d}}
\newcommand{\proj}{\,\mathrm{proj}}
\newcommand{\ran}{\,\mathrm{ran}}
\newcommand{\derive}[1]{\,\frac{\dd}{\dd #1}}
\newcommand{\diff}{\frac{\mathrm{d}y}{\mathrm{d}x}}
\newcommand{\pdiff}[2]{\,\frac{\partial#1}{\partial#2}}
\newcommand{\indre}[2]{\,\left\langle #1, #2\right\rangle}
\newcommand{\norm}[1]{\,\left\|#1\right\|}
\newcommand{\tr}[1]{\,\mathrm{tr}\left(#1\right)}
\newcommand{\inverse}{\,^{-1}}
\newcommand{\hessianto}[1]{\,
	\begin{pmatrix}
		\frac{\partial^2 #1}{\partial x^2} & \frac{\partial^2 #1}{\partial x\partial y}\\
		\frac{\partial^2 #1}{\partial y\partial x} & \frac{\partial^2 #1}{\partial y^2}
	\end{pmatrix}
}
\newcommand{\hessiantre}[1]{\,
	\begin{pmatrix}
		\frac{\partial^2 #1}{\partial x^2} & \frac{\partial^2 #1}{\partial x\partial y} & \frac{\partial^2 #1}{\partial x \partial z}\\
		\frac{\partial^2 #1}{\partial y\partial x} & \frac{\partial^2 #1}{\partial y^2} & \frac{\partial^2 #1}{\partial y \partial z}\\
		\frac{\partial^2 #1}{\partial z\partial y} & \frac{\partial^2 #1}{\partial z\partial y} & \frac{\partial^2 #1}{\partial z^2}
	\end{pmatrix}
}
\newcommand{\corr}[2]{\,\rho(#1, #2)}
\newcommand{\cov}[2]{\,\mathrm{Cov}(#1, #2)}
\newcommand{\sd}{\,\mathrm{SD}}
\newcommand{\prob}[1]{\,P\left(#1\right)}
\newcommand{\var}[1]{\, \mathrm{Var}\left(#1\right)}
\newcommand{\expected}[1]{\,E\left(#1\right)}
\newcommand{\confidence}[3]{\,\left[#1 - #2_{\alpha/2} #3, #1 + #2_{\alpha/2} #3\right]}
\newcommand{\mean}[4]{\,\frac{1}{#4}\sum_{#2 = #3}^{#4} #1_{#2}}
\newcommand{\sVar}[4]{\,\frac{1}{#4 - 1}\sum_{#2 = #3}^{#4} \left(#1_{#2} - \bar{#1}\right)^2}
\newcommand{\tTest}[4]{\,\frac{#1 - #2}{#3/\sqrt{#4}}}

\newcommand{\fref}[1]{\hyperref[#1]{Figur \eqref{#1}}}
\newcommand{\tref}[1]{\hyperref[#1]{Tabell \eqref{#1}}}
\newcommand{\eref}[1]{\hyperref[#1]{Likning \eqref{#1}}}
\newcommand{\rowop}[1]{\,\overset{#1}{\sim}}

\makeatletter
\renewcommand\thesubsection{Oppgave \@arabic\c@subsection)}
\renewcommand\thesubsubsection{\@alph\c@subsubsection)}
\makeatother

%%% HEADERS & FOOTERS
\usepackage{fancyhdr} % This should be set AFTER setting up the page geometry
\pagestyle{fancy} % options: empty , plain , fancy
\renewcommand{\headrulewidth}{0pt} % customise the layout...
\lhead{}\chead{}\rhead{}
\lfoot{}\cfoot{\thepage}\rfoot{}

%%% SECTION TITLE APPEARANCE
\usepackage{sectsty}
\allsectionsfont{\sffamily\mdseries\upshape} % (See the fntguide.pdf for font help)
% (This matches ConTeXt defaults)

%%% ToC (table of contents) APPEARANCE
\usepackage[nottoc,notlof,notlot]{tocbibind} % Put the bibliography in the ToC
\usepackage[titles,subfigure]{tocloft} % Alter the style of the Table of Contents
\renewcommand{\cftsecfont}{\rmfamily\mdseries\upshape}
\renewcommand{\cftsecpagefont}{\rmfamily\mdseries\upshape} % No bold!
\renewcommand\theenumi{\alph{enumi}}

%\makeatletter
%\renewcommand\thesection{}
%\renewcommand\thesubsection{}
%\makeatother

\theoremstyle{plain}
\newtheorem{example}{Eksempel}[section]
\newtheorem{theorem}{Teorem}[section]
\newtheorem*{theorem*}{Teorem}
\newtheorem{corollary}[theorem]{Corollary}
\newtheorem{lemma}[theorem]{Lemma}
\newtheorem{proposition}[theorem]{Proposition}
\newtheorem{axiom}[theorem]{Axiom}
\theoremstyle{definition}
\newtheorem{definition}[theorem]{Definition}
\theoremstyle{remark}
\newtheorem{remark}[theorem]{Remark}
%%% END Article customizations

%%% The "real" document content comes below...
\begin{document}
	
\lstset{language=Matlab,%
	%basicstyle=\color{red},
	breaklines=true,%
	morekeywords={matlab2tikz},
	keywordstyle=\color{blue},%
	morekeywords=[2]{1}, keywordstyle=[2]{\color{black}},
	identifierstyle=\color{black},%
	stringstyle=\color{mylilas},
	commentstyle=\color{mygreen},%
	showstringspaces=false,%without this there will be a symbol in the places where there is a space
	numbers=left,%
	numberstyle={\tiny \color{black}},% size of the numbers
	numbersep=9pt, % this defines how far the numbers are from the text
	emph=[1]{for,end,break, plot},emphstyle=[1]\color{blue}, %some words to emphasise
	%emph=[2]{word1,word2}, emphstyle=[2]{style},    
}

\begin{titlepage}

\newcommand{\HRule}{\rule{\linewidth}{0.5mm}} % Defines a new command for the horizontal lines, change thickness here

\center % Center everything on the page

%----------------------------------------------------------------------------------------
%	HEADING SECTIONS
%----------------------------------------------------------------------------------------

\textsc{\LARGE OsloMet}\\[1.5cm] % Name of your university/college
\textsc{\Large Ingeniørfag - Data}\\[0.5cm] % Major heading such as course name
\textsc{\large DAPE2000 - Matematikk 2000}\\[0.5cm] % Minor heading such as course title

%----------------------------------------------------------------------------------------
%	TITLE SECTION
%----------------------------------------------------------------------------------------

\HRule \\[0.4cm]
{ \huge \bfseries Samling av løsninger til \\ tidligere eksamener}\\[0.4cm] % Title of your document
\HRule \\[1.5cm]

%----------------------------------------------------------------------------------------
%	AUTHOR SECTION
%----------------------------------------------------------------------------------------

\begin{minipage}{0.4\textwidth}
	\begin{flushleft} \large
		\emph{Løst av:}\\
		Uy Quoc \textsc{Nguyen} (s341864) % Your name
	\end{flushleft}
\end{minipage}
~
\begin{minipage}{0.4\textwidth}
	\begin{flushright} \large
		%\emph{Innleveringsfrist:} \\
		%25. Mars 2020
		 % Supervisor's Name
	\end{flushright}
\end{minipage}\\[4cm]

% If you don't want a supervisor, uncomment the two lines below and remove the section above
%\Large \emph{Author:}\\
%John \textsc{Smith}\\[3cm] % Your name

%----------------------------------------------------------------------------------------
%	DATE SECTION
%----------------------------------------------------------------------------------------

%{\large \today}\\[3cm] % Date, change the \today to a set date if you want to be precise

%----------------------------------------------------------------------------------------
%	LOGO SECTION
%----------------------------------------------------------------------------------------

%\includegraphics[width=8cm]{Logo}\\[1cm] % Include a department/university logo - this will require the graphicx package

%----------------------------------------------------------------------------------------

\vfill % Fill the rest of the page with whitespace
\end{titlepage}
\newpage
\setcounter{tocdepth}{1}
\tableofcontents
\newpage
\eksamen{2019 (Kont)}
\oppgave
\deloppgave
For at $f(x)$ skal være en sannsynlighetsstetthet må den oppfylle følgende kritere:
\begin{likning}
	\int_{-\infty}^\infty f(x) \dd x = 1
\end{likning}
med den oppgitte $f(x)$ fra oppgaven har vi at:
\begin{likning}
	\int_{0}^2 \frac{1}{2}x + c \dd x = \left. \frac{1}{4}x^2 + cx \right|_0^2 = 1 + 2c = 1
\end{likning}
løser vi for $c$ får vi at $c = 0$.

\deloppgave
\begin{utregning}
	P(X < 1) = \int_0^1 \frac{1}{2}x \dd x = \left. \frac{1}{4}x^2 \right|_0^1 = \frac{1}{4}
\end{utregning}

\deloppgave
Per derfinisjon har vi at $P(X < 3/2 | X < 1) = P(X < 3/2 \cap X < 1)/P(X < 1)$. Siden $3/2 > 1$ så må $P(X < 3/2 \cap X < 1) = P(X < 1)$. Altså har vi at $P(X < 3/2 | X < 1) = 1$.

\deloppgave
Per definisjon på forventningsverdi $E(X)$:
\begin{likning}
	E(X) = \int_{-\infty}^\infty xf(x) \dd x = \int_0^2 \frac{1}{2}x^2 \dd x = \left. \frac{1}{6}x^3 \right|_0^2 = \frac{4}{3}
\end{likning}

\oppgave
\deloppgave
Sannsynligheten $P(XY \geq 1)$ blir oppfylt for $X = 1$ og $Y = 1$ eller $Y = 2$. Dette gir derfor $P(XY geq 1) = P(1, 1) + P(1, 2) = 0.35$.

\deloppgave
Bruker definisjon av forventningsverdi:
\begin{utregning}
	\mu_X &=& 0\cdot 0.4 + 1\cdot 0.6 = 0.6\\
	\mu_Y &=& 0\cdot 0.3 + 1\cdot 0.4 + 2\cdot 0.3 = 1
\end{utregning}

\deloppgave
Korrelasjonen er definert ved:
\begin{likning}
	\corr{X}{Y} = \frac{\cov{X}{Y}}{\sigma_X\sigma_Y}
\end{likning}
og kovariansen mellom $X$ og $Y$ er gitt ved:
\begin{likning}
	\cov{X}{Y} = E(XY) - \mu_X\mu_Y
\end{likning}
og igjen har vi at
\begin{likning}
	E(XY) = \sum_i\sum_j ijP(x_i, y_i) = 1\cdot 1\cdot P(1,1) + 1\cdot 2\cdot P(1, 2) = 0.6
\end{likning}
Med verdiene som vi fant tidligere har vi at $\rho(X, Y) = 0$.

\deloppgave
Variablene $X$ og $Y$ er ikke uavhengige fordi $P(0, 0) = 0.5$ mens $P(X = 0)P(Y = 0) = 0.12$ (i.e. $P(X, Y) \neq P(X)P(Y)$). 

\oppgave
\deloppgave
Dersom $\lambda = 10$ (ms$\inverse$) så kan vi forvente å motta $1.5\lambda = 15$ på 1.5 ms. Siden $X$ er poisson fordelt så kan vi finne $P(X = 20)$ ved:
\begin{likning}
	P(X = 20) = \frac{15^20}{20!} e^{-15} \approx 0.04181
\end{likning}

\deloppgave
La $T$ være ventetiden før neste datapakke ankommer svitsjen. Det vi nå ønsker å finne er $P(T > t)$ hvor $t = 0.3$. Ved komplement regelen har vi at $P(T > t) = 1 - P(T \leq t)$. Fra forelesning fant vi ut at $T$ i en poissonprosess er eksponentialtfordelt, så vi har at $P(T \leq t) = 1 - e^{-\lambda t}$. Dette gir oss $P(T > t) = e^{-\lambda t} \approx 0.0498$.

\oppgave
Siden vi skal finne en $99\%$ konfidensinterval trenger vi først å finne $z_{\alpha/2}$. Her må $\alpha = 1 - 0.99 = 0.01 \implies \alpha/2 = 0.005$. Slår vi opp i tabellen får vi at $z_{0.005} = 2.576$. Konfidensintervallen for $\mu$ blir derfor 
$$
\left[\bar{X} - z_{\alpha/2}\dfrac{\sigma}{\sqrt{n}}, \bar{X} + z_{\alpha/2}\dfrac{\sigma}{\sqrt{n}}\right]
$$ 
Med $\bar{X} = 17.5$, $\sigma = 3.45$ og $n = 5$ så har vi at det er $99\%$ sikkert at $\mu \in \left[13.526, 21.474\right]$

\oppgave
Oppgaven tyder på at den stokatistiske variabelen $X$ er binomisk fordelt med sannsynligheten $p = 2.84\cdot 10^{-9}$. Men, siden vi trenger bare å finne første gangen vedkommende inntreffer et vellykket knekket passord så blir det naturlig å at $X \sim \mathrm{Geom}(p)$. En geometrisk sannsynlighetsfordeling har en sannsynlighetstetthet oppgitt ved $P(X = x) = p(1-p)^{x-1}$. Forventningsverdien $E(X)$ til en geometrisk distribusjon er oppgitt som $E(X) = 1/p$. Dette kan vi bevise ved å bruke definisjonen for forventningsverdi:
\begin{utregning}
	E(X) = \sum_{x = 1}^\infty xP(X = x) = \sum_{x = 1}^\infty xp(1 - p)^{x-1}
\end{utregning}
Siden $p$ er en konstant kan vi faktorisere den ut av summasjonen:
\begin{utregning}
	\sum_{x = 1}^\infty xp(1 - p)^{x-1} = p\sum_{x = 1}^\infty x(1-p)^{x-1}
\end{utregning}
Siden $1 - p < 1$ så kan vi betrakte følgende geoemtrisk delsum:
\begin{utregning}
	\sum_{x = 1}^\infty (1-p)^x = \frac{1}{1 - (1-p)} - 1
\end{utregning}
Vi utfører en variabel bytte og la $\omega = 1-p$ da har vi at
\begin{utregning}
	\sum_{x = 1}^\infty \omega^x = \frac{1}{1-\omega} - 1
\end{utregning}
Vi deriverer begge sider med hensyn på $\omega$ og får:
\begin{utregning}
	\sum_{x = 1}^\infty x\omega^{x-1} = \frac{1}{(1-\omega)^2}
\end{utregning}
Siden vi har $\omega = 1 - p$ så har vi derfor at
\begin{utregning}
	\sum_{x = 1}^\infty x(1-p)^{x-1} = \frac{1}{p^2}
\end{utregning}
Forventningsverdien $E(X)$ er derfor
\begin{likning}
	E(X) = p\sum_{x = 1}^\infty x(1-p)^{x-1} = \frac{1}{p}
\end{likning}
Bruker vi verdien $p = 2.84\cdot 10^{-9}$ får vi at $E(X) = 352112676.1$

\oppgave
\deloppgave
Siden $B$ har to distinkte egenverdier $\lambda_1 \neq \lambda_2$ så vet vi at $B$ kan diagonaliseres og kan skrive som $B = PDP^{-1}$ hvor matrisen $D$ er gitt som:
\begin{likning}
	D = \begin{pmatrix}
		\lambda_1 & 0\\
		0 & \lambda_2
	\end{pmatrix}
\end{likning}
Dette gir igjen da at
\begin{utregning}
	B^2 - 3B + 2I = PD^2P\inverse - 3PDP\inverse + 2I = 0
\end{utregning}
slik at
\begin{utregning}
	D^2P\inverse - 3DP\inverse + 2P\inverse I &=& P\inverse 0\\
	D^2 - 3D + 2P\inverse I P &=& 0P\\
	D^2 - 3D + 2I &=& 0
\end{utregning}
Dette gir oss følgende likningssystem:
\begin{utregning}
	\lambda_1^2 - 3\lambda_2 + 2 = (\lambda_1 - 2)(\lambda_1 - 1) = 0\\
	\lambda_2^2 - 3\lambda_2 + 2 = (\lambda_2 - 2)(\lambda_2 - 1) = 0
\end{utregning}
Egenverdiene for $B$ er derfor $\lambda_1 = 2 \implies \lambda_2 = 1$ eller $\lambda_1 = 1 \implies \lambda_2 = 2$.

\deloppgave
Her har vi to måter å løse på. Enten så kan vi prøve å finne egenverdiene til $A$ og sammenligne de med de påståtte egenverdiene til $A$ eller så kan vi bruke at $\lambda_1 + \lambda_2 = \tr{A} = 4$ og $\lambda_1\lambda_2 = \det(A) = -8$.

Dersom man skal finne egenverdiene til $A$ så kan vi starte med å løse $\det(A - \lambda I) = 0$ som gir oss:
\begin{utregning}
	\det(A - \lambda I) &=& (2-\lambda)^2 - 12\\
	&=& (2 - \lambda)^2 - (2\sqrt{3})^2\\
	&=& (2 - \lambda - 2\sqrt{3})(2 - \lambda + 2\sqrt{3}) = 0
\end{utregning}
Herifra er det enkelt å se at $\lambda = 2 \pm 2\sqrt{3} = 2(1 \pm \sqrt{3})$. Som stemmer overens med det som har blitt oppgitt i oppgaven.

Dersom vi skulle gå for den andre metoden har vi at
\begin{utregning}
	\lambda_1 + \lambda_2 = 2 - 2\sqrt{3} + 2+2\sqrt{3} = 4 
\end{utregning}
og
\begin{utregning}
	\lambda_1\lambda_2 = (2-2\sqrt{3})(2+2\sqrt{3}) = 2^2 - 12 = -8
\end{utregning}
Som også stemmer overens med det vi har fått oppgitt.

\deloppgave
Nå som vi har egenverdiene for $A$ så kan vi finne egenvektorene ved å løse $A - \lambda I$.

La $\lambda_1 = 2(1 - \sqrt{3})$ da har vi
\begin{utregning}
	\begin{pmatrix}
		2 - 2(1 - \sqrt{3}) & 6\\
		2 & 2 - 2(1-\sqrt{3}) 
	\end{pmatrix}
	\sim
	\begin{pmatrix}
		\sqrt{3} & 3\\
		1 & \sqrt{3}
	\end{pmatrix}
	\sim
	\begin{pmatrix}
		1 & \sqrt{3}\\
		0 & 0
	\end{pmatrix}
\end{utregning}
Dette gir oss egenvektoren $\vec{v}_1 = (-\sqrt{3}, 1)$. 

For $\lambda_2 = 2(1 + \sqrt{3})$ kan vi følge de samme stegene:
\begin{utregning}
	\begin{pmatrix}
		2 - 2(1 + \sqrt{3}) & 6\\
		2 & 2 - 2(1 +\sqrt{3}) 
	\end{pmatrix}
	\sim
	\begin{pmatrix}
		-\sqrt{3} & 3\\
		1 & -\sqrt{3}
	\end{pmatrix}
	\sim
	\begin{pmatrix}
		1 & -\sqrt{3}\\
		0 & 0
	\end{pmatrix}
\end{utregning}
Som gir oss $\vec{v}_2 = (\sqrt{3}, 1)$. 

\deloppgave
La $\vec{x}' = (x', y')$. Da kan vi uttrykke systemet som $\vec{x}' = A\vec{x}$ hvor $A$ er matrisen oppgitt over. Den generelle løsningen for et slikt system av differensiallikninger er $\vec{x} = c_1\vec{x}_1 + c_2\vec{x}_2$ der $c_1$ og $c_2$ er skalarer og
\begin{utregning}
	\vec{x}_1 &=& \vec{v}_1e^{\lambda_1 t}\\
	\vec{x}_2 &=& \vec{v}_2e^{\lambda_2 t}
\end{utregning}
Her er $\lambda$ egenverdiene til $A$ og $\vec{v}$ er egenvektorene til $A$. Dette gir oss løsningen:
\begin{utregning}
	\vec{x} &=& c_1\begin{pmatrix} -\sqrt{3}\\1 \end{pmatrix}e^{2(1-\sqrt{3})t} + c_2\begin{pmatrix}
		\sqrt{3} \\ 1
	\end{pmatrix} e^{2(1+\sqrt{3})t}
\end{utregning}
Gitt initialbetingelsen $\vec{x}(0) = (1,1)$ så har vi at
\begin{utregning}
	\begin{pmatrix}
		-\sqrt{3} & \sqrt{3}\\
		1 & 1
	\end{pmatrix}
	\begin{pmatrix}
		c_1\\c_2
	\end{pmatrix} &=&
	\begin{pmatrix}
		0\\1
	\end{pmatrix}\\
	\begin{pmatrix}
		c_1\\ c_2
	\end{pmatrix}
	&=&
	-\frac{\sqrt{3}}{6}
	\begin{pmatrix}
		1 & -\sqrt{3}\\
		-1 & -\sqrt{3}
	\end{pmatrix}
	\begin{pmatrix}
		0\\1
	\end{pmatrix}
	=
	\frac{1}{2}
	\begin{pmatrix}
		1\\
		1
	\end{pmatrix}
\end{utregning}
Som gir oss den partikulære løsningen:
\begin{likning}
	\vec{x}_p = \frac{1}{2}
	\begin{pmatrix}
		-\sqrt{3}\\1
	\end{pmatrix} e^{2(1-\sqrt{3})t}
	+
	\frac{1}{2}
	\begin{pmatrix}
		\sqrt{3}\\1
	\end{pmatrix} e^{2(1+\sqrt{3})t}
\end{likning}

\oppgave
\deloppgave
For denne rekken kan vi bare bruke en enkel divergenstest:
\begin{likning}
	\lim_{n \to \infty} \frac{5n}{8n + 3n^{2/3}} = \frac{5}{8}
\end{likning}
Dette tilsier at rekken divergerer.

\deloppgave
For denne rekken er det lett å tenke seg å bruke integral-testen. Problemet her er at det ikke er så enkel å integrere funksjonen $1/\ln x$. Vi velger derfor å bruke sammenligningstesten. Betrakt nå $a_n = \dfrac{1}{n\ln n}$. La nå $b_n = \dfrac{1}{\ln n}$. Her er det tydelig at $a_n \leq b_n$ fordi $n\ln n \geq \ln n$ for alle verdier $n \geq 1$. Ved å bruke integraltesten på $a_n$ finner vi ut at den divergerer:
\begin{utregning}
	\int_2^\infty \frac{1}{x\ln x} \dd x = \left. \ln x \right|_2^\infty \to \infty
\end{utregning}
Siden $a_n \leq b_n$ og $\sum a_n$ divergerer så må også $\sum b_n$ divergere, i.e. rekken oppgitt i oppgaven divergerer.

\deloppgave
Betrakt først den geometriske rekken:
\begin{likning}
	\sum_{n = 1}^\infty x^n = \frac{1}{1 - x} - 1 \qquad |x| < 1
\end{likning}
Deriverer vi begge sider får vi da
\begin{utregning}
	\sum_{n = 1}^\infty nx^{n-1} = \frac{1}{(1 - x)^2}
\end{utregning}

\oppgave
\deloppgave
Ett kritisk punkt $\vec{x}_0$ oppfyller $\nabla f(\vec{x}_0) = \vec{0}$. Spesifikt for den oppgitte $f(x, y)$ har vi at
\begin{utregning}
	\pdiff{f}{x} &=& \frac{1}{5}x-10xe^{-(x^2 + y^2)} = \frac{1}{5}x\left(1 - 50e^{-(x^2 + y^2)}\right)\\
	\pdiff{f}{y} &=& \frac{1}{5}y - 10ye^{-(x^2 + y^2)} = \frac{1}{5}y\left(1 - 50e^{-(x^2 + y^2)}\right)
\end{utregning}
Vi har derfor at $\nabla f(0,0) = 0$, så $(0,0)$ er et kritisk punkt. Videre har vi også at for alle punkter som $(x, y)$ som oppfyller $x^2 + y^2 = \ln 50$ så vil $1 - 50e^{-\ln 50} = 0$. Dette betyr også at $\nabla f(x, y) = 0$ som betyr at disse også er kritiske punkter.

\deloppgave
For å unngå rot, så lar vi $\theta(0,0) = x^2 + y^2$. De kritiske punktene vi skal sjekke for nå er når $(x, y) = (0,0)$ og $\theta(x, y) = \ln 50$. Betrakt først:
\begin{utregning}
	\frac{\partial^2 f}{\partial x^2} &=& \frac{1}{5} - 10e^{-\theta}\left(1 + 2x^2\right)
\end{utregning}
For punktet $(0,0)$ har vi at $\partial^2f/\partial x^2 = -49/5$. Siden $\partial^2f/\partial x^2 < 0$ så ser vi på et potensielt maksimumspunkt. Vi finner også andre derivasjonen av de andre variablene:
\begin{utregning}
	\frac{\partial^2 f}{\partial x\partial y} &=& 20xye^{-\theta}\\
	\frac{\partial^2 f}{\partial y\partial x} &=& \frac{\partial^2 f}{\partial x\partial y}\\
	\frac{\partial^2 f}{\partial y^2} &=& \frac{1}{5} - 10e^{-\theta}(1+2y^2) 
\end{utregning}
Dette gir følgende Hessian matrise $\mathcal{H}$:
\begin{likning}
	\mathcal{H} = \hessianto{f} = \begin{pmatrix}
		-49/5 & 0\\
		0 & -49/5
	\end{pmatrix}
\end{likning}
for punktet $(0,0)$ og dermed $\det\mathcal{H} > 0$. Vi har derfor et maksimumspunkt for punktet $(0,0)$. 

For $\theta = \ln 50$. Har vi
\begin{utregning}
	\frac{\partial^2f}{\partial x^2} = \frac{2}{5}x^2
\end{utregning}
Her er det helt klart at $\partial^2f/\partial x^2 > 0$ for alle $x \neq 0$. Hessian matrisen her $\mathcal{H}$ blir dermed:
\begin{likning}
	\mathcal{H} = \frac{2}{5}\begin{pmatrix}
		x^2 & xy\\
		xy & y^2
	\end{pmatrix}
\end{likning}
Herifra er det enkelt å se at diskriminanten $\det\mathcal H = 0$. Det vil si at for $\theta = \ln 50$ så har vi ingen konklusjon via denne testen.

Vi får heller sammenligne verdiene. Vi har $f(0,0) = 5$ mens for $\theta = \ln 50$ har vi
\begin{likning}
	f(x_\theta, y_\theta) = \frac{1}{10} + \frac{1}{10}\ln 50 = \frac{1}{10}\left(1 + \ln 50\right)
\end{likning}
Herifra har vi at $f(0,0) > f(x_\theta, y_\theta)$. Siden punktet $(0,0)$ var funnet til å være en maksimumspunkt så må $(x_\theta, y_\theta)$ være minimumspunktene.

Siden funksjonen er definert for alle punkter $x^2 + y^2 \leq 100$. Vi sjekker derfor for randpunktene som oppfyller $x^2 + y^2 = 100$. Når $x^2 + y^2 = 100$ har vi at
\begin{likning}
	f(x_{100}, y_{100}) = \frac{5}{e^{100}} + 10 \approx 10
\end{likning}

Vi har derfor at $(0,0)$ er et lokalt maksimumspunkt, $(x_\theta, y_\theta)$ er et globalt og lokalt minimumspunkt. Mens, $(x_{100}, y_{100})$ er et globalt maksimumspunkt.
\clearpage
\eksamen{2019}
\oppgave
Legg merke til at
\begin{utregning}
	\prob{X = 0} + \prob{X = 1} + \prob{X = 2} = 1 \quad \implies \quad \prob{X = 2} = 0.200
\end{utregning}
\deloppgave
Sannsynligheten $\prob{X < 1} = \prob{X = 0} = 0.250$.

\deloppgave
Her har vi fra betinget sannynlighet:
\begin{likning}
	\prob{X > 0 | X > 1} = \frac{\prob{X > 0 \cap X \geq 1}}{\prob{X \geq 1}}
\end{likning}
Siden $\{1, 2\} \cap \{1, 2\} = \{1, 2\}$ så har vi
\begin{likning}
	\prob{X > 0 | X > 1} = \frac{\prob{X \geq 1}}{\prob{X \geq 1}} = 1
\end{likning}

\deloppgave
Per definisjon av forventningsverdi:
\begin{likning}
	\mu_X = 1\cdot \prob{X = 1} + 2\prob{X = 2} = 0.950
\end{likning}

\deloppgave
Per definisjon av variasjon $\var{X}$:
\begin{utregning}
	\var{X} &=& E\left(X^2\right) - \left[E\left(X\right)\right]^2\\
	&=& 1\cdot P(X = 1) + 4\cdot P(X = 2) - 0.950^2 = 0.4475
\end{utregning}
Dette gir $\sigma_X = \sqrt{\var{X}} = 0.6690$

\deloppgave
Vi skal nå finne forventningsverdien $E\left((Y-X)^2\right)$:
\begin{utregning}
	E\left((Y-X)^2\right) &=& E\left(X^2 + Y^2 - 2XY\right)\\
	&=& E\left(X^2\right) + E\left(Y^2\right) - 2E\left(XY\right)
\end{utregning}
Vi har også fått oppgitt at $\mu_Y = 1.80$ og $\sigma_Y = 1.20$. Siden
\begin{utregning}
	\because \var{Y} &=& E(Y^2) - \mu_Y^2\\
	\therefore \expected{Y^2} &=& \sigma_Y^2 + \mu_Y^2
\end{utregning}
Dette gir
\begin{likning}
	\expected{\left(Y - X)^2\right)} = \sigma_X^2 + \mu_X^2 + \sigma_Y^2 + \mu_Y^2 - 2\expected{XY}
\end{likning}
Videre har vi
\begin{utregning}
	\because \corr{X}{Y} &=& \frac{\cov{X}{Y}}{\sigma_X\sigma_Y}\\
	\therefore \cov{X}{Y} &=& \sigma_X\sigma_Y\corr{X}{Y}
\end{utregning}
Siden
\begin{utregning}
	\because \cov{X}{Y} &=& \expected{XY} - \mu_X\mu_Y\\
	\therefore \expected{XY} &=& \sigma_X\sigma_Y\corr{X}{Y} + \mu_X\mu_Y 
\end{utregning}
Dette gir oss:
\begin{utregning}
	\expected{\left(Y - X\right)^2} &=& \sigma_X^2 + \mu_X^2 + \sigma_Y^2 + \mu_Y^2 - 2\sigma_X\sigma_Y\corr{X}{Y} - 2\mu_X\mu_Y\\
	&=& \left(\mu_Y - \mu_X\right)^2 + \left(\sigma_Y - \sigma_X\right)^2 + 2\sigma_X\sigma_Y\left(1 - \corr{X}{Y}\right) \approx 1.722
\end{utregning}

\oppgave
\deloppgave
Siden $X \sim \mathrm{bin}(n, p)$ så er $\expected{X} = np$. Dette gir at
\begin{likning}
	\expected{\hat{p}} = \expected{\frac{X}{n}} = \frac{1}{n}\expected{X} = p
\end{likning}

\deloppgave
For en binomisk distribusjon så har vi at $\var{X} = np(1-p)$. Dette gir at
\begin{likning}
	\var{\hat{p}} = \frac{1}{n^2}\var{X} = \frac{p}{n}(1-p)
\end{likning}

\deloppgave
For at en estimator skal kunne bli klassifisert som en god estimator så krever vi at estimatoren er forventningsrett. Siden $\expected{\hat{p}} = p$ så er dette kravet oppfylt. Videre ønsker vi at variansen skal være så lite som mulig. Dette kravet handler mer om sammenligningen mellom to foreslåtte estimatorer og er ikke noe vi kan undersøke her. Tilslutt ønsker vi at variansen til estimatoren skal gå mot null når $n\to\infty$. Per utregning vist i forrige oppgave så ser vi at dette er tilfellet. Den oppgitte $\hat{p}$ er derfor en god estimator for $p$.

\deloppgave
Gitt $X = 354$ så har vi at $\hat{p} = 354/500$. En $95\%$ ($\alpha = 0.05$) konfidensinterval for $p$ er derfor:
\begin{utregning}
	\confidence{\hat{p}}{z}{\sqrt{\frac{\hat{p}}{n}(1-\hat{p})}} = \left[0.6681, 0.7479\right]
\end{utregning}

\deloppgave
Grunnen til at konfidensintervallet oppgitt ovenfor er gyldig er på grunn av sentral grenseteoremet. I henhold til sentral grenseteoremet er forutsetningene at $n\hat{p}(1-\hat{p}) > 5$. I vårt tilfellet så er $n\hat{p}(1-\hat{p}) = 103.368$ som er mer enn tilstrekkelig for å oppfylle det kravet.

\oppgave
For denne spesifikke problemstillingen velger vi å utføre en $T$-test fordi $\sigma$ er ukjent. La
\begin{utregning}
	H_0 \colon&& \mu \leq 15\\
	H_1 \colon&& \mu > 15
\end{utregning}
\begin{utregning}
	\bar{x} &=& \mean{x}{i}{1}{5} = 18.52
\end{utregning}
og
\begin{utregning}
	s^2 &=& \sVar{x}{i}{1}{5} = 18.177
\end{utregning}
Dette gir oss en
\begin{utregning}
	t &=& \tTest{\bar{x}}{15}{s}{5} = 1.846
\end{utregning}
Med en signifikansnivå $\alpha = 0.10$ så har vi
\begin{utregning}
	t_{\alpha}^{4} = 1.533
\end{utregning}
Siden $t > t_\alpha^4$ så forkaster vi $H_0$.

\oppgave
\deloppgave
Siden matrisen $A$ er en $2\times 2$-matrise så gjelder det at den karakteristiske likningen er
\begin{utregning}
	\det(A-\lambda I) &=& \lambda^2 - \tr{A}\lambda + \det(A)\\
	&=& \lambda^2 - 2\lambda - 2 = 0
\end{utregning}
Ved $abc$-formellen får vi at
\begin{likning}
	\lambda = \frac{2 \pm 2\sqrt{3}}{2} = 1\pm \sqrt{3}
\end{likning}
Egenverdiene til $A$ er derfor $\lambda_1 = 1-\sqrt{3}$ og $\lambda_2 = 1 + \sqrt{3}$.

For $\lambda_1$ så kan vi dedusere egenvektoren $\vec{v}_1$ til $A$:
\begin{utregning}
	\begin{pmatrix}
		\sqrt{3} & 3\\
		1 & \sqrt{3}
	\end{pmatrix}
	\sim
	\begin{pmatrix}
		1 & \sqrt{3}\\
		0 & 0
	\end{pmatrix}
\end{utregning}
som gir egenvektoren $\vec{v}_1 = (-\sqrt{3}, 1)$. Ved samme resonnement får vi $\vec{v}_2 = (\sqrt{3}, 1)$. $A$ kan derfor diagonaliseres:
\begin{likning}
	A = \frac{\sqrt{3}}{6}
	\begin{pmatrix}
	-\sqrt{3} & \sqrt{3}\\
	1 & 1
	\end{pmatrix}
	\begin{pmatrix}
		1 - \sqrt{3} & 0\\
		0 & 1 + \sqrt{3}
	\end{pmatrix}
	\begin{pmatrix}
		-1 & \sqrt{3}\\
		1 & \sqrt{3}
	\end{pmatrix}
\end{likning}

\deloppgave
La $D$ og $P$ være henholdsvis
\begin{utregning}
	D &=& \begin{pmatrix}
	1 - \sqrt{3} & 0\\
	0 & 1 + \sqrt{3}
\end{pmatrix}\\
	P &=& \begin{pmatrix}
		-\sqrt{3} & \sqrt{3}\\
		1 & 1
	\end{pmatrix}
\end{utregning}
Da har vi at
\begin{utregning}
	B &=& P\left(D^2 + 3D + 2I\right)P\inverse
\end{utregning}
Vi ser fra uttrykket ovenfor at egenverdiene til $B$ er
\begin{utregning}
	\lambda_1 &=& (1-\sqrt{3})^2 + 3(1-\sqrt{3}) + 2 = 9-5\sqrt{3}\\
	\lambda_2 &=& (1+\sqrt{3})^2 + 3(1+\sqrt{3}) + 2 = 9+5\sqrt{3}
\end{utregning}
Dette gir diagonalisering av $B$:
\begin{likning}
	B = \frac{\sqrt{3}}{6}
	\begin{pmatrix}
		-\sqrt{3} & \sqrt{3}\\
		1 & 1
	\end{pmatrix}
	\begin{pmatrix}
		9-5\sqrt{3} & 0\\
		0 & 9+5\sqrt{3}
	\end{pmatrix}
	\begin{pmatrix}
		-1 & \sqrt{3}\\
		1 & \sqrt{3}
	\end{pmatrix}
\end{likning}

\deloppgave
Observer at vi kan uttrykke systemet som $\vec{x}' = A\vec{x}$. Dette gir den generelle løsningen
\begin{likning}
	\vec{x} = c_1\vec{v}_1e^{\lambda_1 t} + c_2\vec{v}_2e^{\lambda_2 t}
\end{likning}
der $\lambda$ og $\vec{v}$ er henholdsvis egenverdiene og egenvektorene til $A$. Vi har derfor
\begin{likning}
	\vec{x} = c_1\begin{pmatrix}-\sqrt{3}\\1\end{pmatrix} e^{(1-\sqrt{3})t} + c_2\begin{pmatrix} \sqrt{3}\\1\end{pmatrix} e^{(1+\sqrt{3})t}
\end{likning}
Gitt initialbetingelsen $\vec{x}(0) = (0, 1)$ så har vi at
\begin{likning}
	\begin{pmatrix}
		-\sqrt{3} & \sqrt{3}\\
		1 & 1
	\end{pmatrix}\begin{pmatrix}
	c_1\\c_2
\end{pmatrix}
= \begin{pmatrix}
	0\\1
\end{pmatrix}
\end{likning}
Løser vi systemet får vi
\begin{likning}
	\begin{pmatrix}
		c_1\\c_2
	\end{pmatrix}
	= \frac{\sqrt{3}}{6}\begin{pmatrix}
		-1 & \sqrt{3}\\
		1 & \sqrt{3}
	\end{pmatrix}
	\begin{pmatrix}
		0\\1
	\end{pmatrix}
	= \frac{\sqrt{3}}{6}
	\begin{pmatrix}
		\sqrt{3}\\
		\sqrt{3}
	\end{pmatrix}
\end{likning}
Dette gir oss den partikulære løsningen:
\begin{likning}
	\vec{x}_p = \frac{1}{2}\begin{pmatrix}-\sqrt{3}\\1\end{pmatrix} e^{(1-\sqrt{3})t} + \frac{1}{2}\begin{pmatrix} \sqrt{3}\\1\end{pmatrix} e^{(1+\sqrt{3})t}
\end{likning}

\oppgave
\deloppgave
Vi kan uttrykke den oppgitte rekken som:
\begin{utregning}
	\sum_{n = 0}^\infty \frac{4^n - 3^n}{5^n} &=& \sum_{n = 0}^\infty \left(\frac{4}{5}\right)^n - \sum_{n = 0}^\infty \left(\frac{3}{5}\right)^n
\end{utregning}
Fra geometriske rekker vet vi at disse rekkene konvergerer til
\begin{utregning}
	\sum_{n = 0}^\infty \frac{4^n - 3^n}{5^n} &=& \frac{1}{1-4/5} - \frac{1}{1-3/5} = \frac{5}{2}
\end{utregning}

\deloppgave
\begin{enumerate}
	\item Ved divergenstesten så har vi at
	\begin{likning}
		\lim_{n \to \infty} \frac{2}{2+\left(8/9\right)^n} = 1
	\end{likning}
	så konkluderer vi med at rekken vil divergere.
	
	\item Vi bruker sammenligningstesten: La $b_n = n^{3/2}$ og $a_n$ være den oppgitte følgen. Da har vi at $0 \leq a_n \leq b_n$. Dersom $\sum b_n$ konvergerer så konvergerer $\sum a_n$. Vi kjenner igjen at $b_n$ er en $p$-rekke med $p < 1$ som tilsier at $b_n$ konvergerer. Dette vil si at den oppgitte rekken vil konvergere.
\end{enumerate}

\deloppgave
La $u = t^2$. Da har vi at
\begin{utregning}
	\sin u &=& \sum_{n = 0}^\infty \left(-1\right)^n\frac{u^{2n+1}}{\left(2n+1\right)!}\\
	\sin t^2 &=& \sum_{n = 0}^\infty \left(-1\right)^n\frac{t^{2(2n+1)}}{\left(2n+1\right)!}\\
	&=& \sum_{n = 0}^\infty \left(-1\right)^n\frac{t^{4n+2}}{\left(2n+1\right)!}\\
	&=& \sum_{n = 0}^\infty \left(-1\right)^n\frac{t^{4n}t^2}{\left(2n+1\right)!}
\end{utregning}
Dette gir at
\begin{utregning}
	\frac{\sin t^2}{t} &=& \sum_{n = 0}^\infty \left(-1\right)^n\frac{t^{4n}t^2}{t\left(2n+1\right)!}\\
	&=& \sum_{n = 0}^\infty \left(-1\right)^n\frac{t^{4n+1}}{\left(2n+1\right)!}
\end{utregning}
Dette vil si at
\begin{utregning}
	\int_0^x \frac{\sin t^2}{t} \dd t &=& \sum_{n = 0}^\infty \frac{\left(-1\right)^n}{\left(2n+1\right)!}\int_0^xt^{4n+1}\dd t\\
	&=& \sum_{n = 0}^\infty \frac{\left(-1\right)^n}{\left(2n+1\right)!}\left[\frac{t^{4n+2}}{4n+2}\right]_0^x\\
	&=& \sum_{n = 0}^\infty \frac{\left(-1\right)^n}{\left(2n+1\right)!}\cdot\frac{x^{2(2n+1)}}{2(2n+1)}
\end{utregning}

\oppgave
Tangentplanet til en funksjonen $g(x)$ er gitt ved $\indre{\nabla g(x, y, z)}{\vec{x} - \vec{x}_0} = 0$ der $\vec{x} = (x, y, z)$ og i dette tilfellet $\vec{x}_0 = (1, 1, 5)$.
\begin{utregning}
	\nabla g(x, y, z) = \begin{pmatrix}
		4x\\4y\\2z
	\end{pmatrix}
\end{utregning}
Dette gir
\begin{utregning}
	\indre{\nabla g(x, y, z)}{\vec{x} - \vec{x}_0} &=& \pdiff{g}{x}(x - x_0) + \pdiff{g}{y}(y-y_0) + \pdiff{g}{z}(z - z_0)\\
	&=& 4(x - 1) + 4(y - 1) + 10(z - 5)\\
	&=& 4x + 4y + 10z -58 \implies 2x + 2y + 5z = 29
\end{utregning}

\clearpage
\eksamen{2018}
\oppgave
Når $T\sim\mathrm{exp(\mu)}$ så er sannsynlighetstettheten gitt ved $f(t) = e^{-t/\mu}/\mu$. Det vil si at generelt
\begin{utregning}
	P(a < T < b) = \frac{1}{\mu}\int_{a}^b e^{-t/\mu}\dd t = e^{-a/\mu} - e^{-b/\mu}
\end{utregning}
Slik at
\deloppgave
\begin{likning}
	P(T > 90) = P(90 < T < \infty) = e^{-6/5} \approx 0.3012
\end{likning}
\deloppgave
\begin{likning}
	P(50 < T < 90) = e^{-2/3} - e^{-6/5} \approx 0.2122
\end{likning}
\deloppgave
\begin{likning}
	P(T\geq 150 | T > 90) = \frac{P(T \geq 150 \cap T > 90)}{P(T > 90)} = \frac{P(T\geq 150)}{P(T > 90)} = \frac{e^{-2}}{e^{-6/5}} \approx 0.4493
\end{likning}

\deloppgave
Vi har allerede fått oppgitt forventningen til populasjonen $\mu = 75$. Vi har da at
\begin{likning}
	\bar{T} = \frac{1}{20}\sum_{n = 1}^{20} T_n
\end{likning}
Og derfor
\begin{likning}
	E(\bar{T}) = \frac{1}{20}E\left(\sum_{n = 1}^{20} T_n\right)
\end{likning}
Siden $T_n$ er uavhengige og derfor
\begin{likning}
	E(\bar{T}) = \frac{1}{20} \sum_{n = 1}^{20} E(T_n)
\end{likning}
Siden hver av de målingene $T_n \sim \mathrm{exp}(\mu)$ så har vi at $E(T_n) = \mu$ for alle $n \in \mathbb Z_{21}\setminus{0}$. Dette gir
\begin{likning}
	E(\bar{T}) = \frac{1}{20}\cdot 20\mu = \mu
\end{likning}

\deloppgave
Per definisjon av $\sd^2(\bar{T})$ så har vi
\begin{utregning}
	\sd^2(\bar{T}) &=& \sd^2\left(\frac{1}{20}\sum_{n=1}^{20} T_n\right)\\
	&=& \frac{1}{400}\sd^2\left(\sum_{n=1}^{20} T_n\right)\\
	&=& \frac{1}{400}\sd^2\sum_{n=1}^{20}\sd^2\left(\bar{T}\right)\\
	&=& \frac{1}{400}\sd^2\sum_{n=1}^{20}\sigma^2\\
	&=& \frac{1}{400}\cdot 20\mu^2 = \frac{\mu^2}{20}
\end{utregning}
Dette gir $\sd(\bar{T}) = \sqrt{\sd^2(\bar{T})} = 16.77$.

\deloppgave
Siden $n = 20$ er betraktlig stor nok kan vi utnytte sentral grenseteoremet og anta at $\bar{T}\sim N\left(E(\bar{T}), \sd(\bar{T})\right)$. Med dette, kan vi derfor finne $P(\bar{T} > 90) = 1 - P(\bar{T}\leq 90)$, hvor
\begin{likning}
	P(\bar{T}\leq 90) = P\left(\frac{\bar{T}-\mu}{\sigma/\sqrt{n}} \leq \frac{90-\mu}{\sigma/\sqrt{n}}\right) = \Phi\left(Z \leq 0.8944\right) = 0.8133 
\end{likning}
Dette gir $P\left(\bar{T} > 90\right) = 0.1867$.

\deloppgave
Dersom $\bar{T} \sim \mathrm{exp}(\mu)$ så må $E(\bar{T}) = \sd(\bar{T})$. Siden vi fant ut at den ikke var det (i.e. $\bar{T}$ er ikke eksponentialfordelt) så kan vi desverre ikke regne ut en eksakt verdi for $\prob{\bar{T} > 90}$ ved å bruke eksponentialfordelingen.

\deloppgave
Null-hypotesen formuleres først ut ifra mistanken til problemstillingen som har blitt presentert. Siden mistanken her er at $\mu > 60$ så blir dette formuleringen for $H_1$. Det naturlige blir da å formulere $H_0$ som en komplement av $H_1$ altså $H_0 = \mu \leq 60$.

\deloppgave
Her utfører vi en $T$-test fordi $\sigma$ er ukjent. Gitt $\bar{t} = 75.3$ og $s = 50.7$ for $n = 30$ målinger så har vi
\begin{likning}
	T = \frac{\bar{t} - \mu_0}{s/\sqrt{n}} = 1.653
\end{likning}
Vi har også fra tabellen at $T_\alpha^{n-1} = 1.311$. Siden $T > T_\alpha^{n-1}$ så må ingeniørene konkludere med en $\alpha = 0.1$ signifikant-nivå at det tilstrekkelig med grunnlag å forkaste $H_0$.

\deloppgave
For en $90\%$ konfidensintervall av en $T$-test er vi ute etter å finne $T_{\alpha/2}^{n-1} = 1.699$. Vi har derfor at konfidensintervallen til $\mu$ er
\begin{likning}
	\left[\bar{t} - T_{\alpha/2}^{n-1}\frac{s}{\sqrt{n}}, \bar{t} + T_{\alpha/2}^{n-1}\frac{s}{\sqrt{n}}\right] = [59.57, 91.03]
\end{likning}

\oppgave
\deloppgave
Egenvierdiene $\lambda$ til en matrise $A$ er gitt ved å løse den karakteristiske likningen $\det(A-\lambda I) = 0$. Siden $A$ er en $2\times 2$-matrise kan vi benytte oss av at $\lambda_1 + \lambda_2 = \tr{A}$ og at $\lambda_1\lambda_2 = \det(A)$. Dette gir oss følgende:
\begin{utregning}
	\lambda_1 + \lambda_2 &=& 0\\
	\lambda_1\lambda_2 &=& -1
\end{utregning}
Løser vi likningen oppgitt ovenfor får vi at $\lambda_1 = \pm 1$. Dette medfølger at $\lambda_2 = \mp 1$. 
\newline

La $\lambda_1 = -1$ og $\lambda_2 = 1$. Egenvektorene $\vec{v}_1$ og $\vec{v}_2$ til $A$ får vi av å løse likningen:
\begin{likning}
	(A-\lambda I)\vec{v} = \vec{0}
\end{likning}
Med $\lambda_1 = -1$ får vi at
\begin{utregning}
	\begin{pmatrix}
		1 & 1\\
		1 & 1
	\end{pmatrix}&\sim&
	\begin{pmatrix}
		1 & 1\\
		0 & 0
	\end{pmatrix}
\end{utregning}
Dette gir oss $\vec{v}_1 = (1, -1)$. Ved samme argumentasjon for $\lambda_2 = 1$:
\begin{utregning}
	\begin{pmatrix}
		-1 & 1\\
		1 & -1
	\end{pmatrix} \sim
	\begin{pmatrix}
		1 & -1\\
		0 & 0
	\end{pmatrix}
\end{utregning}
får vi at $\vec{v}_2 = (1, 1)$.

Med egenverdiene og egenvektorene funnet over kan vi danne matrisen $P$ og matrisen $D$ slik at vi kan uttrykke $A$ som $A = PDP\inverse$. Hvor
\begin{likning}
	P = \begin{pmatrix}
		1 & 1\\
		-1 & 1
	\end{pmatrix} \implies P\inverse = \frac{1}{2}\begin{pmatrix}
	1 & -1\\
	1 & 1
\end{pmatrix}
\end{likning}
og 
\begin{likning}
	D = \begin{pmatrix}
		-1 & 0\\
		0 & 1
	\end{pmatrix}
\end{likning}
Dette gir oss
\begin{likning}
	A = \frac{1}{2}
	\begin{pmatrix}
		1 & 1\\
		-1 & 1
	\end{pmatrix}
	\begin{pmatrix}
		-1 & 0\\
		0 & 1
	\end{pmatrix}
	\begin{pmatrix}
		1 & -1\\
		1 & 1
	\end{pmatrix}
\end{likning}

Vi kan nå diagonalisere $B = A^7 + A^5 + I$ som følger:
\begin{likning}
	B = P\left(D^7 + D^5 + I\right)P\inverse
\end{likning}
egenverdiene til $B$ blir her $\lambda_1 = (-1)^7 + (-1)^5 + 1 = -1$ og $\lambda_2 = 3$. Dette gir diagonaliseringen av $B$:
\begin{likning}
	B = 
	\frac{1}{2}
	\begin{pmatrix}
		1 & 1\\
		-1 & 1
	\end{pmatrix}
	\begin{pmatrix}
		-1 & 0\\
		0 & 3
	\end{pmatrix}
	\begin{pmatrix}
		1 & -1\\
		1 & 1
	\end{pmatrix}
\end{likning}

\deloppgave
Omskriver vi systemet får vi 
\begin{utregning}
	x' &=& 0x + 1y\\
	y' &=& 1x + 0y
\end{utregning}
Her ser vi helt klart at vi kan uttrykke likningssystemet som
\begin{likning}
	\vec{x}' = A\vec{x}
\end{likning}
der $A$ er matrisen oppgitt i oppgaven.

Vi identifiserer at vi har et tilfelle av en homogent likning. For å løse dette må vi finne egenverdiene $\lambda$ og egenvektorene $\vec{v}$ til $A$. Dette har vi gjort tidligere. Systemet vil derfor ha en generell løsning gitt ved
\begin{likning}
	\vec{x} = c_1\vec{x}_1 + c_2\vec{x}_2
\end{likning}
hvor $c_1$ og $c_2$ er skalarer og 
\begin{utregning}
	\vec{x}_1 &=& \vec{v}_1 e^{\lambda_1 t}\\
	\vec{x}_2 &=& \vec{v}_2 e^{\lambda_2 t}
\end{utregning}
Med egenverdiene for $\lambda_1$ og $\lambda_2$ og henholdsvis egenvektorene $\vec{v}_1$ og $\vec{v}_2$ som vi tidligere fant for matrise $A$ har vi derfor at løsningen for systemet av differensiallikninger er gitt ved
\begin{utregning}
	\vec{x} &=& c_1\begin{pmatrix}1\\-1\end{pmatrix}e^{-t} + c_2 \begin{pmatrix}
		1 \\ 1
	\end{pmatrix} e^t
\end{utregning}
Ved initialbetingelsen $\vec{x}(0) = (1 ,1)$ har vi derfor at
\begin{utregning}
	\begin{pmatrix}
		1 & 1\\
		-1 & 1
	\end{pmatrix}\begin{pmatrix}
	c_1\\
	c_2
\end{pmatrix} &=& \begin{pmatrix}
	1\\1
\end{pmatrix}\\
\begin{pmatrix}
	c_1\\c_2
\end{pmatrix}
&=& \frac{1}{2}\begin{pmatrix}
	1 & -1\\
	1 & 1
\end{pmatrix}\begin{pmatrix}
	1\\1
\end{pmatrix} = \begin{pmatrix}
0\\1
\end{pmatrix}
\end{utregning}
Den partikulære løsningen er derfor
\begin{likning}
	\vec{x}_p = \begin{pmatrix}
		1\\1
	\end{pmatrix} e^t
\end{likning}

\oppgave
\deloppgave
\begin{enumerate}
	\item Rekken er en geometrisk rekke med felles forholdet $1/e < 1$. Dette tyder på at rekken vil konvergere mot:
	\begin{likning}
		\sum_{n = 1}^{\infty} \left(\frac{1}{e}\right)^n = \frac{1}{1-1/e} - 1 = \frac{1}{e-1}
	\end{likning}
	
	\item Her kan vi benytte oss av forholdstesten: La $a_n = (-3)^n/n$ da har vi at 
	\begin{utregning}
		\frac{a_{n+1}}{a_n} &=& \frac{(-3)^{n+1}}{n+1}\cdot \frac{n}{(-3)^n} = -\frac{3n}{n+1}
	\end{utregning}
	Vi får herifra at $\lim_{n\to\infty} \left|-3n/(n+1)\right| = 3 > 1$. Vi konkluderer derfor at rekken divergerer.
\end{enumerate}

\deloppgave
Vi vet at Maclaurin rekken til $e^{t}$ er gitt ved:
\begin{likning}
	e^t = \sum_{n = 0}^\infty \frac{t^n}{n!}
\end{likning}
Dette gir oss at
\begin{utregning}
	f(x) = 2xe^{x^2} &=& 2x\sum_{n = 0}^\infty \frac{x^{2n}}{n!}\\
	&=& \sum_{n = 0}^\infty \frac{2}{n!}x^{2n+1}
\end{utregning}
Konvergensradien til følgende rekke er 
\begin{utregning}
	R = \lim_{n \to \infty} \left| \frac{2}{n!} \cdot \frac{(n+1)!}{2} \right| = \lim_{n \to \infty} \left| n+1 \right| \to \infty
\end{utregning}

\oppgave
\deloppgave
Tangentplanet er gitt ved $\indre{\nabla f}{\Delta \vec{x}}$. La $f(x, y, z) = x^2 + y^3 - z^2 = 0$. Da har vi at
\begin{utregning}
	\indre{\nabla f}{\Delta \vec{x}} &=& 2x_0(x - x_0) + 3y_0^2(y - y_0) - 2z_0(z - z_0)\\
	&=& 2(x - 1) + 3(y - 1) - 4(z - 2)\\
	&=& 2x + 3y - 4z = -3
\end{utregning}

\deloppgave
Retningsvektoren måler hvor mye $g$ øker når vi beveger oss i retningen $\vec{u}$. Retningsvektoren er gitt ved:
\begin{likning}
	D_{\vec{u}} = \indre{\nabla g}{\frac{\vec{u}}{\norm{\vec{u}}}}
\end{likning}
Spesifikt for indreproduktet av et vektorrom så har du også at
\begin{likning}
	D_{\vec{u}} = \norm{\nabla g}\norm{\frac{\vec{u}}{\norm{\vec{u}}}}\cos\theta = \norm{\nabla g}\cos\theta
\end{likning}
Siden $\nabla g$ alltid peker mot der $g$ vokser fortest, la $\nabla g\parallel\vec{u}$. Da vil $\norm{\nabla g}$ være den største verdien av den retningsderiverte. Videre har vi også at
\begin{utregning}
	\norm{\nabla g}^2 = \indre{\nabla g}{\nabla g}
\end{utregning}
Med
\begin{utregning}
	\pdiff{g}{x} &=& -2xe^{-(x^2 + y^2)}\\
	\pdiff{g}{y} &=& -2ye^{-(x^2 + y^2)}
\end{utregning}
så har vi
\begin{likning}
	\norm{\nabla g}^2 = 4x^2e^{-2(x^2+y^2)} + 4y^2e^{-2(x^2 + y^2)} = 4(x^2 + y^2)e^{-2(x^2+y^2)}
\end{likning}
La $\theta = x^2 + y^2$ da har vi
\begin{likning}
	\norm{\nabla g}^2 = 4\theta e^{-2\theta}
\end{likning}
som gir
\begin{likning}
	\frac{\dd }{\dd \theta} 4\theta e^{-2\theta} = 4\left(1 - 2\theta^2\right)e^{-2\theta}
\end{likning}
Vi har at $\theta^2 = 1/2$ er et kritisk punkt for funksjonen over. Andre derivasjonstesten
\begin{likning}
	\frac{\dd^2}{\dd\theta^2} 4\theta e^{-2\theta} = 16(\theta - 1)e^{-2\theta}
\end{likning}
er negativ for $\theta = 1/\sqrt{2}$ som tyder på at $\theta^2 = 1/2$ er et topp punkt for $4\theta e^{-2\theta}$. Det vil si at alle punktene $(x, y)$ som $x^2 + y^2 = 1/2$ danner en sirkel om origo med radius $\sqrt{2}/2$ gir størst mulig verdi for den retningsderiverte av $g(x, y)$.

\clearpage
\eksamen{2016}
\statistikk
\setcounter{subsection}{2}
\oppgave
\deloppgave
Det man må merke seg her er at det har ikke blitt oppgitt at målingene er normalfordelt eller ikke. Vanligvis må dette være et krav for at vi kan gjennomføre en $T$-test eller at antall målinger er tilstrekkelig for å benytte seg av sentralgrenseteoremet. Men, ihenhold til den informasjonen så gjennomfører vi avlikevel testen. La hypotese-testen vår være formulert ved:
\begin{utregning}
	H_0\colon && \mu \leq 15\\
	H_1\colon && \mu > 15
\end{utregning}
Her har vi at
\begin{utregning}
	\bar{x} &=& \mean{x}{i}{1}{5} = 18.04\\
	s^2 &=& \sVar{x}{i}{1}{5} = 23.133
\end{utregning}
Dette gir oss
\begin{utregning}
	t = \tTest{\bar{x}}{15}{s}{5} = 1.4133
\end{utregning}
Ved et signifikants-nivå $\alpha = 0.1$ har vi at $t_{0.1}^{4} = 1.533$. Siden $t < t_{\alpha}^{n-1}$ så må vi konkludere med å behold null-hypotesen $H_0$.

\deloppgave
Som nevnt tidligere; ingen steder i oppgaveteksten ble det avklart at målingene var normalfordelte. Dersom målingene ikke er normalfordelte krever vi at $n \geq 30$ for at hypotese-testen skal være gyldig. Uten noen antagelser så er ikke hypotesetesten utført i denne oppgaven gyldig.
%\oppgave
\deloppgave
\begin{enumerate}
	\item Følgende rekke kan uttrykkes som
	\begin{likning}
		\sum_{n = 1}^\infty \left(\frac{1}{n} - \frac{1}{n+1}\right) = \sum_{n=1}^\infty \frac{1}{n(n+1)}
	\end{likning}
	Her kan vi bruke sammenligningstesten: La $b_n = \dfrac{1}{n^2}$ og la derfor $a_n = \dfrac{1}{n(n+1)}$. Vi har nå at $b_n \geq 0$ og $a_n \geq 0$. Videre har vi også at $b_n \geq a_n$ for alle $n \geq 1$. Vi kjenner igjen at $b_n$ er en $p$-rekke og konvergerer siden $p \geq 1$. Per sammenligningstesten så konvergerer også $a_n$, altså, rekken konvergerer.
	
	\item Her kan vi bruke forholdstesten: Med den gitte rekken har vi følgende:
	\begin{likning}
		a_n = \frac{n+1}{n+2}\frac{1}{2^n} \implies a_{n+1} = \frac{n+2}{n+3}\frac{1}{2^{n+1}} 
	\end{likning}
	Det er enkelt å se at
	\begin{likning}
		\lim_{n\to\infty}\left|\frac{a_{n+1}}{a_n}\right| = \lim_{n\to\infty}\left|\frac{(n+2)^2}{2(n+3)(n+1)}\right| = \frac{1}{2}
	\end{likning}
	Vi har derfor per forholdstesten at rekken konvergerer absolutt.
	
	\item Her har vi en altererende følge. Dette tyder fort på at det er Dirichlets test vi må bruke. La $b_n = (-1)^n$ da har vi at $\left|\sum_{n = 1}^N b_n\right| \leq 1$ for alle $N$. La $a_n = 1/n$. Da har vi at $\sum_{n = 1}^{\infty} a_n \geq 0$, $a_i \geq a_{i+1}$ for alle $i\in\mathbb{N}_{>0}$ og $\lim_{n\to\infty} a_n = 0$. Da sier Dirichlets test at rekken $\sum_{n=1}^\infty b_na_n$ konvergerer. Så, den oppgitte rekken konvergerer.
\end{enumerate}

\deloppgave
For en rekke $\sum_{n=0}^\infty b_n(x-c)^n$ så er konvergensradien $R$ definert som
\begin{likning}
	R\equiv \lim_{n\to\infty}\left|\frac{b_n}{b_{n+1}}\right|
\end{likning}
For den oppgitte rekken i oppgaven har vi at $b_n = (n+1)^2$ og $c = 0$. Dette gir
\begin{likning}
	R = \lim_{n\to\infty}\left|\frac{n+1}{n+2}\right|^2 = 1
\end{likning}
Konvergensradien $R$ for den oppgitte rekken er derfor $R = 1$.

\deloppgave
Vi skal finne Maclaurin rekken til $f(x) = x^2e^x + x$. Vi vet tidligere at Maclaurin rekken til $e^x$ er gitt ved:
\begin{likning}
	e^x = \sum_{n = 0}^\infty \frac{x^n}{n!}
\end{likning}
Det vil si at vi kan uttrykke $f(x)$ som 
\begin{likning}
	f(x) = x + x^2\sum_{n = 0}^\infty \frac{x^n}{n!} = x + \sum_{n = 0}^\infty \frac{x^{n+2}}{n!}
\end{likning}

\oppgave
\deloppgave
La $f\colon \mathbb R^n \to \mathbb R$ og la $\vec{u}$ være en vektor. Da er den retningsderiverte $D_{\vec{u}} f(\vec{r})$ av $f$ på punktet $\vec{r}\in\mathbb R^n$ i retningen til $\vec{u}$ gitt ved
\begin{likning}
	D_{\vec{u}}f(\vec{r}) = \indre{\nabla f(\vec{r})}{\frac{\vec{u}}{\norm{\vec{u}}}} 
\end{likning}
Gitt $g(x, y, z) = x^2 + y^2 + z^2$ så har vi:
\begin{utregning}
	\pdiff{g}{x} &=& 2x\\
	\pdiff{g}{y} &=& 2y\\
	\pdiff{g}{z} &=& 2z
\end{utregning}
med $\vec{r} = (1, 1, 1)$ så har vi $\nabla f(\vec{r}) = (2, 2, 2)$. Videre har vi at
\begin{likning}
	\frac{\vec{u}}{\norm{\vec{u}}} = \begin{pmatrix}
		0\\1/\sqrt{2}\\1/\sqrt{2}
	\end{pmatrix}
\end{likning}
Dette gir oss
\begin{likning}
	D_{\vec{u}}g(\vec{r}) = 2\sqrt{2}
\end{likning}

\deloppgave
La $\vec{x} = (x, y, z)$ og la $\vec{r} = (1, 1, 1)$. Vi fant tidligere at $\nabla g(\vec{r}) = (2, 2, 2)$. Dette gir oss tangentplanet for nivåkurven $g(x, y, z) = 3$ gitt ved
\begin{utregning}
	\indre{\nabla g(\vec{r})}{\vec{x}-\vec{r}} &=& 0\\
	2(x - 1) + 2(y - 1) + 2(z - 1) &=& 0\\
	x + y + z &=& 3
\end{utregning}

\deloppgave
Dersom et vilkårlig punkt $\vec{p}$ er et kritisk punkt for en funksjon $f(x, y z)$ så må det være slikt at $\nabla f(\vec{p}) = \vec{0}$. Dette ser vi tydelig fra tidligere beregnelse av $\nabla g(x, y z) = (2x, 2y, 2x)$. Det er klart her at $(0, 0, 0)$ er et kritisk punkt fordi $\nabla g(0, 0, 0) = (0, 0, 0)$. Det er også enkelt å se at dette er den eneste kritiske punktet siden det ikke er noen andre punkter som oppfyller kriteriet.

Så, den eneste kritiske punktet vi har er $(0, 0, 0)$ for å identifisere hva slags kritisk punkt dette er bruker vi den 2. partielle derivative testen: Vi finner først Hessian matrisen $\mathcal{H}$ av $g$:
\begin{likning}
	\mathcal{H}(x, y, z) = \hessiantre{g} = \begin{pmatrix}
		2 & 0 & 0\\
		0 & 2 & 0\\
		0 & 0 & 2
	\end{pmatrix}
\end{likning}
Diskriminanten er her $D\equiv \det\left(\mathcal{H}(x, y, z)\right) = 8$. Vi ser også ifra matrisen at $\partial^2 g/\partial x^2 > 0$. Dette tilsier per andre partialle derivasjons testen at dette er et lokalt minimum.

\clearpage

\end{document}