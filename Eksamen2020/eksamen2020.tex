\documentclass[12pt, a4paper,norsk]{article}
\usepackage[utf8]{inputenc}
\usepackage[T1]{fontenc,url}
\usepackage{babel,textcomp}
\usepackage{amsmath, amssymb}
\usepackage{polynom}
\usepackage[parfill]{parskip}


% ------------------------------------------------------------
% Side formatering
\usepackage{fancyhdr}
\lhead{DAPE2000 - Eksamen 2020 Høst}
\rhead{Kandidatnr: 574}
\usepackage{lastpage}
\cfoot{Side \thepage\hspace{.5pt} av \pageref{LastPage}}
% ------------------------------------------------------------

% ------------------------------------------------------------
% Seksjons formatering
\makeatletter
\renewcommand\thesection{Oppgave \@arabic\c@section)}
\renewcommand\thesubsection{\@alph\c@subsection)}
\makeatother
\newcommand{\oppgave}{\,\section{}}
\newcommand{\deloppgave}{\subsection{}}
\renewcommand{\labelenumi}{\roman{enumi})}
% ------------------------------------------------------------

% ------------------------------------------------------------
% Egne kommandoer for matematiske uttrykk:
\newcommand{\pdiff}[2][]{\frac{\partial#1}{\partial #2}}
\newcommand{\integral}[4]{\int_{#1}^{#2} #3 \mathrm{d} #4}
\newcommand{\series}[3]{\sum_{#1}^{#2} #3}
\newcommand{\eigval}[2][]{\lambda_{#2}^{#1}}
\newcommand{\eigvec}[3][]{\vec{#2}_{#3}^{#1}}

\newcommand{\confidence}[3]{\left[#1 - #2_{\alpha/2} #3, #1 + #2_{\alpha/2} #3\right]}
\newcommand{\mean}[4]{\frac{1}{#4}\sum_{#2 = #3}^{#4} #1_{#2}}
\newcommand{\sVar}[4]{\frac{1}{#4 - 1}\sum_{#2 = #3}^{#4} \left(#1_{#2} - \bar{#1}\right)^2}
\newcommand{\tTest}[4]{\frac{#1 - #2}{#3/\sqrt{#4}}}
\newcommand{\corr}[2]{\rho\left(#1, #2\right)}
\newcommand{\cov}[2]{\mathrm{Cov}\left(#1, #2\right)}
\newcommand{\sd}[1]{\mathrm{SD}\left(#1\right)}
\newcommand{\prob}[1]{P\left(#1\right)}
\newcommand{\var}[1]{\mathrm{Var}\left(#1\right)}
\newcommand{\expected}[1]{E\left(#1\right)}
\newcommand{\binomDist}[2]{\sim Bin\left(#1, #2\right)}
\newcommand{\normalDist}[2]{\sim N\left(#1, #2\right)}
\newcommand{\score}[3]{\frac{#1 - #2}{#3}}

\newcommand{\dd}{\mathrm{d}}
\newcommand{\indre}[2]{\left\langle #1, #2\right\rangle}
\newcommand{\norm}[1]{\left\|#1\right\|}
\newcommand{\tr}[1]{\mathrm{tr}\left(#1\right)}
\newcommand{\inverse}{^{-1}}
\newcommand{\hessianto}[1][f]{
	\begin{pmatrix}
		\frac{\partial^2 #1}{\partial x^2} & \frac{\partial^2 #1}{\partial x\partial y}\\
		\frac{\partial^2 #1}{\partial y\partial x} & \frac{\partial^2 #1}{\partial y^2}
	\end{pmatrix}
}
\newcommand{\hessiantre}[1][f]{
	\begin{pmatrix}
		\frac{\partial^2 #1}{\partial x^2} & \frac{\partial^2 #1}{\partial x\partial y} & \frac{\partial^2 #1}{\partial x \partial z}\\
		\frac{\partial^2 #1}{\partial y\partial x} & \frac{\partial^2 #1}{\partial y^2} & \frac{\partial^2 #1}{\partial y \partial z}\\
		\frac{\partial^2 #1}{\partial z\partial y} & \frac{\partial^2 #1}{\partial z\partial y} & \frac{\partial^2 #1}{\partial z^2}
	\end{pmatrix}
}

\newcommand{\tabell}[1]{\footnote{
	Institutt for matematiske fag NTNU,
	"Tabeller og formler i statistikk", 
	6. opplag 2011, side #1.}}
\newcommand{\formler}[1]{\footnote{
	Karl Rottmann, "Matematisk Formelsamling", 
	6. opplag 2011, side #1.}}

% ------------------------------------------------------------

\title
{
	DAPE2000: \\
	Matematikk 2000 med statistikk\\
	Eksamen 2020 Høst
}
\author{Kandidatnr: 574\\Antall sider: \pageref{LastPage}}
\date{10. Desember 2020}

\begin{document}
	\maketitle
	\pagenumbering{gobble}
	\clearpage
	\pagestyle{fancy}
	\pagenumbering{arabic}
	\oppgave
	Siden summen av alle sannsynlighetene må bli $1$ så er det enkelt å se her at $\prob{X = 2} = 0.2$.
	
	\deloppgave
	$$
	\prob{0 < X < 2} = \prob{X = 1} = 0.07
	$$
	
	\deloppgave
	Per definisjon på betinget sannsynlighet så har vi at 
	$$
	\prob{X \leq 2 | X > 1} = \frac{\prob{X \leq 2 \cap X > 1}}{\prob{X > 1}}
	$$
	Siden $\prob{X \leq 2 \cap X > 1} = \prob{X > 1} = \prob{X = 2}$. Så gir det oss at $\prob{X \leq 2 | X > 1} = 1$.
	
	\deloppgave
	Per definisjon av forventningsverdi
	$$
	\expected{X} = \mu_X = \sum_{i = 0}^\infty x_i\prob{X = x_i} = 1\cdot 0.07 + 2\cdot 0.2 = 0.47
	$$
	
	\deloppgave
	Per definisjon av standardavviket har vi at
	$$
	\var{X} = \expected{X^2} - \mu_X^2
	$$
	Hvor
	$$
	\expected{X^2} = \sum_{i = 0}^\infty x_i^2P(X = x_i) = 1\cdot 0.07 + 4\cdot 0.2 = 0.87
	$$
	Dette gir at $\var{X} = \sigma_X^2 = 0.6491$ som gir igjen at standardavviket til $X$ er $\sigma_X \approx 0.81$.
	
	\deloppgave
	Fra definisjon av forventningsverdi så har vi at
	$$
	\expected{g\left(X\right)} = \sum_{x\in X} g(x)\prob{X = x}
	$$
	La $g\left(X\right) = a + bX + cX^2$ vi har fra lineæriteten til summen at
	\begin{align*}
	\sum_{x\in X} \left(a + bx + cx^2\right)\prob{X = x} &=
	\sum_{x\in X} a\prob{X = x} + \sum_{x\in X} bx\prob{X = x} + \sum_{x\in X} cx^2\prob{X = x}\\
	&= a\sum_{x\in X} \prob{X = x} + b\sum_{x\in X}^\infty x\prob{X = x} + c\sum_{x\in X} x^2\prob{X = x}
	\end{align*}
	Videre har vi også at $\sum_{x\in X}\prob{X = x} = 1$. Itillegg til at $\expected{X} = \sum_{x\in X}x\prob{X = x}$ og $\expected{X^2} = \sum_{x\in X} x^2\prob{X = x}$ som tilslutt gir oss:
	$$
	\expected{a + bX + cX^2} = a + b\expected{X} + c\expected{X^2}
	$$
	
	\oppgave
	\deloppgave
	La $X$ være antall fotoner som blir produsert. Vi har at $X \sim Poisson(\lambda)$ hvor $\lambda = 0.2(\text{ns})\inverse$. Siden $X$ er poissonfordelt har vi også at sannsynlighetstettheten til $X$ er gitt ved
	$$
	\prob{X = x} = \frac{\left(\lambda t\right)^x}{x!}e^{-\lambda t}
	$$
	der $\lambda t = 10$. Vi skal finne $P(X > 12) = 1 - P(X \leq 12)$. Slår vi opp i tabellen\tabell{20} får vi at $P(X\leq 12) = 0.7916$. Dette gir oss $P(X > 12) = 0.2084$.
	
	\deloppgave
	For en poissonfordeling har vi at $\expected{X} = \var{X}$. Dette gir at $\sigma = \sqrt{\lambda t}$ der $\lambda = 0.2(\text{ns})\inverse$ og $t = 50\text{ns}$. Så, standardavviket til $X$ iløpet av $50$ns er $\sigma \approx 3.2$.
	
	\deloppgave
	Ventetiden $T$ er eksponentialfordelt med $\lambda = 0.2(\text{ns})\inverse$ og har derfor sannsynlighetstettheten
	$$
	\prob{T = t} = \lambda e^{-\lambda x}
	$$
	Vi skal nå finne $\prob{T > 10} = 1 - \prob{T \leq 10}$. Vi har da at
	$$
	\prob{T \leq 10} = \lambda\int_{0}^{10} e^{-\lambda t} \dd t= \left. -e^{-\lambda t}\right|_{0}^{10} = 1 - e^{-2}
	$$
	Slik at $\prob{T > 10} = e^{-2} \approx 0.1353$.
	
	\deloppgave
	Vi har at siden $\lambda$ er ukjent så vil også både forventningsverdien og variansen av en poissonfordeling også være ukjent. Siden $n \geq 30$ kan vi per sentralgrenseteoremet anta at målingene er normalfordelt. Vi bruker derfor en $T$-interval med de oppgitte verdiene for $\bar{x} = 8.94$, $s = 3.27$ og $n = 30$. Videre har vi $\alpha = 0.05$ og derfor fra tabellen\tabell{4} at $t_{\alpha/2}^{n-1} = 2.045$. Konfidensintervallet vårt er derfor:
	$$
	\confidence{\bar{x}}{t^{n-1}}{\frac{s}{\sqrt{n}}} = [7.7191, 10.1609]
	$$
	
	Videre har vi at estimatet $\hat{\lambda} = X/t$ må være forventningsrett. Vi vet også at for en poissonfordeling så er $\expected{X} = \lambda t$ slik at $\lambda = \expected{X}{t}$. Siden vi har gjennomsnittet $\bar{x}$ så kan vi estimere $\hat{\lambda}$ til å være $\hat{\lambda} = \bar{x}/t = 0.1788(\text{ns})\inverse$ ($t = 50\text{ns}$).
	\clearpage
	\deloppgave
	Vi gjennomfører hypotese-testen med en $T$-test med følgende hypoteser:
	\begin{align}
		H_0\colon & \mu \geq 10\\
		H_1\colon & \mu < 10
	\end{align}
	Vi har fra resultatene av $\bar{x}$ og $s$ at 
	$$
	t = \score{\bar{x}}{10}{s/\sqrt{30}} = -1.775
	$$
	Fra tabellen\tabell{4} har vi at $t_{\alpha}^{n-1} = 2.462$. Siden $t > -t_{\alpha}^{n-1}$ så har vi ikke tilstrekkelig med grunnlag for å forkaste $H_0$. Altså, forskeren har ikke tilstrekkelig bevis for at  forventet antall fotoner $\mu$ som apparatet produserer i løpet av 50
	nanosekunder er mindre enn 10.
	
	\oppgave
	\deloppgave
	Egenverdiene til matrisen $A$ er løsningene til den karakteristiske likningen $\det(A - \lambda I) = \lambda^2 - 5\lambda = \lambda(\lambda - 5) = 0$. Egenverdiene til $A$ er derfor $\lambda_1 = 0$ og $\lambda_2 = 5$. 
	
	Egenvektorene til $A$ finner vi ved å løse $A - \lambda I = 0$ for $\lambda = \lambda_1$ og $\lambda = \lambda_2$.
	
	La $\lambda = \lambda_1$ da har vi at
	$$
	\begin{pmatrix}
		4 & -2\\
		-2 & 1
	\end{pmatrix} \sim
	\begin{pmatrix}
		2 & -1\\
		0 & 0
	\end{pmatrix}
	$$
	som gir oss egenvektoren $\vec{v}_1 = \begin{pmatrix}1 & 2\end{pmatrix}^T$. 
	
	Tilsvarende for $\lambda = \lambda_2$ har vi at
	$$
	\begin{pmatrix}
		-1 & -2\\
		-2 & -4
	\end{pmatrix}\sim
	\begin{pmatrix}
		1 & 2\\
		0 & 0
	\end{pmatrix}
	$$
	som gir oss egenvektoren $\vec{v}_2 = \begin{pmatrix} -2 & 1 \end{pmatrix}^T$
	
	\deloppgave
	Med resultatene vi har fra forrige oppgave har vi at matrisen $D$ er gitt ved
	$$
	D = \begin{pmatrix}
		0 & 0\\
		0 & 5
	\end{pmatrix}
	$$
	og matrisen $P$
	$$
	P = \begin{pmatrix}
		1 & -2\\
		2 & 1
	\end{pmatrix}
	\implies
	P\inverse = \frac{1}{5}\begin{pmatrix}
		1 & 2\\
		-2 & 1
	\end{pmatrix}
	$$
	Vi kan derfor uttrykke $A$ som $A = PDP\inverse$.
	
	\deloppgave
	La $\vec{x}(t) = (x(t), y(t))$. Da har vi at systemet av differensiallikninger oppgitt i oppgaven kan uttrykkes som $\vec{x}'(t) = A\vec{x}(t)$. Dette har da den generelle løsningen
	$$
	\vec{x}(t) = c_1\begin{pmatrix}
		1\\2
	\end{pmatrix} + c_2\begin{pmatrix}
	-2\\1
	\end{pmatrix}e^{5t}
	$$
	Ved initialbetingelsen $\vec{x}(0) = (0, 1)$ har vi at den partikulære løsningen $\vec{x}_p(t)$ er $\vec{x}(t)$ gitt ovenfor der $c_1$ og $c_2$ er
	$$
	\begin{pmatrix}
		c_1\\c_2
	\end{pmatrix}
	= P\inverse \begin{pmatrix}
		0\\1
	\end{pmatrix} = \frac{1}{5}\begin{pmatrix}
	1 & 2\\
	-2 & 1
	\end{pmatrix}\begin{pmatrix}
	0\\1
	\end{pmatrix} = \frac{1}{5}\begin{pmatrix}
	2\\1
	\end{pmatrix}
	$$
	Den partikulære løsningen for differensiallikningen som tilfredstiller initialbetingelsen er derfor
	$$
	\vec{x}_p(t) = \frac{2}{5}\begin{pmatrix}
		1\\2
	\end{pmatrix} + \frac{1}{5}\begin{pmatrix}
		-2\\1
	\end{pmatrix}e^{5t}
	$$
	\clearpage
	\oppgave
	\deloppgave
	Siden $\ln(1) = 0$ så kan vi se bort ifra første leddet i taylor-rekken. Det vil si at vi evaluerer summen mellom $1$ og $\infty$ for denne rekken. Videre har vi at
	\begin{align*}
		f^{(1)}(x) &= \frac{1}{x} \implies f^{(1)}(1) = 1\\
		f^{(2)}(x) &= -\frac{1}{x^2} \implies f^{(2)}(1) = -1\\
		f^{(3)}(x) &= \frac{2}{x^3} \implies f^{(3)}(1) = 2\\
		f^{(4)}(x) &= -\frac{6}{x^4} \implies f^{(4)}(1) = -6\\
		&\vdots\\
		f^{(n)}(x) &= \left(-1\right)^{n+1}\frac{1}{x^{n}} \implies f^{(n)}(1) = \left(-1\right)^{n+1}(n-1)!
	\end{align*}
	Dette gir at
	$$
	\sum_{n = 1}^\infty \frac{\left(-1\right)^{n+1}(n-1)!}{n!}(x-1)^n = \sum_{n = 1}^\infty \frac{\left(-1\right)^{n+1}}{n}(x-1)^n
	$$
	
	\deloppgave
	Konvergensradien $R$ for an potensrekke gitt som $\sum b_n(x - c)^n$ er gitt ved
	$$
	R \equiv \lim_{n\to\infty}\left|\frac{b_n}{b_{n+1}}\right|
	$$
	Konvergensradien $R$ til rekken oppgitt er derfor
	\begin{align*}
	R &= \lim_{n\to\infty}\left|\frac{\left(-1\right)^{n+1}}{n}\cdot\frac{n+1}{(-1)\left(-1\right)^{n+1}}\right|\\
	&= \lim_{n\to\infty}\frac{n+1}{n} \\
	&= \lim_{n\to\infty}1 + \frac{1}{n} = 1
	\end{align*}
	Siden $R = 1$ så vet vi at rekken konvergerer absolutt når $|x - c| < R$ med andre ord $c - R < x < c + R$. Med $c = 1$. Så da har vi at rekken oppgitt konvergerer absolutt når $0 < x < 2$.
	
	\deloppgave
	Hvis vi nå lar $x = 2$ så har vi at
	$$
	\sum_{n = 1}^\infty \frac{\left(-1\right)^{n+1}}{n}(x-1)^n = \sum_{n = 1}^\infty \frac{\left(-1\right)^{n+1}}{n}(1)^n = \sum_{n = 1}^\infty \frac{\left(-1\right)^{n+1}}{n}
	$$
	Her kan vi sjekke konvergens til rekken ved å bruke Dirichlet's test. La $b_n = (-1)^{n+1}$ Da har vi at $\left|\sum_{n = 1}^N b_n\right| \leq 1$ for alle $N$. La videre $a_n = 1/n$ da har vi at $a_n \geq 0$ for alle $n\in\mathbb N$. Vi har også at $a_{n+1} \leq a_n$ og at $\lim_{n\to\infty} a_n = 0$. Siden alle kravene er tilfredstilt så sier Dirichlet's test så konvergerer rekken $\sum_{n = 1}^\infty b_na_n$ konvergerer. Altså rekken
	$$
	\sum_{n = 1}^\infty \frac{\left(-1\right)^{n+1}}{n}(x-1)^n
	$$
	konvergerer når $x = 2$.
	
	\deloppgave
	Vi har fra den geometriske rekken at
	$$
	\frac{1}{1 - r} = \sum_{n = 0}^\infty r^n
	$$
	når $|r| < 1$. La nå $r = 1 - x$ da har vi at
	$$
	\frac{1}{x} = \sum_{n = 0}^\infty (1 - x)^n
	$$
	Integrerer vi på begge sider får vi
	$$
	\ln{x} = \sum_{n = 0}^\infty \frac{-1}{n+1}(1 - x)^{n+1}
	$$
	Videre har vi at $(1 - x)^{n+1} = (-1)^{n+1}(x - 1)^{n+1}$ slik at vi får
	$$
	\ln{x} = \sum_{n = 0}^\infty \frac{(-1)^{n+2}}{n+1}(x - 1)^{n+1}
	$$
	Lar vi nå summen gå fra $n = 1$ istedenfor $n = 0$ så har vi at
	$$
	\ln(x) = \sum_{n = 1}^\infty \frac{(-1)^{n+1}}{n}(x - 1)^{n}
	$$
	
	\oppgave
	La $\vec{r} = \begin{pmatrix} x & y \end{pmatrix}^T$ da vil tangenten til nivåkurven $g(\vec{r}) = 9$ i det oppgitte punktet $\vec{r}_0 = \begin{pmatrix} 2 & 1\end{pmatrix}^T$ være gitt ved
	$$
	\nabla g(\vec{r}_0)\cdot \vec{r} = \nabla g(\vec{r})\cdot \vec{r}_0
	$$
	Med
	$$
	\pdiff[g]{x} = 4x \qquad \text{og} \qquad \pdiff[g]{y} = 2y
	$$
	Så har vi at $\nabla g(\vec{r}_0) = \begin{pmatrix} 8 & 2\end{pmatrix}^T$. Dette gir oss tangenten $8x + 2y = 18$ som kan forenkles til $4x + y = 9$.
\end{document}