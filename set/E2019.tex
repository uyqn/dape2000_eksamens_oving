\eksamen{2019}
\oppgave
Legg merke til at
\begin{utregning}
	\prob{X = 0} + \prob{X = 1} + \prob{X = 2} = 1 \quad \implies \quad \prob{X = 2} = 0.200
\end{utregning}
\deloppgave
Sannsynligheten $\prob{X < 1} = \prob{X = 0} = 0.250$.

\deloppgave
Her har vi fra betinget sannynlighet:
\begin{likning}
	\prob{X > 0 | X > 1} = \frac{\prob{X > 0 \cap X \geq 1}}{\prob{X \geq 1}}
\end{likning}
Siden $\{1, 2\} \cap \{1, 2\} = \{1, 2\}$ så har vi
\begin{likning}
	\prob{X > 0 | X > 1} = \frac{\prob{X \geq 1}}{\prob{X \geq 1}} = 1
\end{likning}

\deloppgave
Per definisjon av forventningsverdi:
\begin{likning}
	\mu_X = 1\cdot \prob{X = 1} + 2\prob{X = 2} = 0.950
\end{likning}

\deloppgave
Per definisjon av variasjon $\var{X}$:
\begin{utregning}
	\var{X} &=& E\left(X^2\right) - \left[E\left(X\right)\right]^2\\
	&=& 1\cdot P(X = 1) + 4\cdot P(X = 2) - 0.950^2 = 0.4475
\end{utregning}
Dette gir $\sigma_X = \sqrt{\var{X}} = 0.6690$

\deloppgave
Vi skal nå finne forventningsverdien $E\left((Y-X)^2\right)$:
\begin{utregning}
	E\left((Y-X)^2\right) &=& E\left(X^2 + Y^2 - 2XY\right)\\
	&=& E\left(X^2\right) + E\left(Y^2\right) - 2E\left(XY\right)
\end{utregning}
Vi har også fått oppgitt at $\mu_Y = 1.80$ og $\sigma_Y = 1.20$. Siden
\begin{utregning}
	\because \var{Y} &=& E(Y^2) - \mu_Y^2\\
	\therefore \expected{Y^2} &=& \sigma_Y^2 + \mu_Y^2
\end{utregning}
Dette gir
\begin{likning}
	\expected{\left(Y - X)^2\right)} = \sigma_X^2 + \mu_X^2 + \sigma_Y^2 + \mu_Y^2 - 2\expected{XY}
\end{likning}
Videre har vi
\begin{utregning}
	\because \corr{X}{Y} &=& \frac{\cov{X}{Y}}{\sigma_X\sigma_Y}\\
	\therefore \cov{X}{Y} &=& \sigma_X\sigma_Y\corr{X}{Y}
\end{utregning}
Siden
\begin{utregning}
	\because \cov{X}{Y} &=& \expected{XY} - \mu_X\mu_Y\\
	\therefore \expected{XY} &=& \sigma_X\sigma_Y\corr{X}{Y} + \mu_X\mu_Y 
\end{utregning}
Dette gir oss:
\begin{utregning}
	\expected{\left(Y - X\right)^2} &=& \sigma_X^2 + \mu_X^2 + \sigma_Y^2 + \mu_Y^2 - 2\sigma_X\sigma_Y\corr{X}{Y} - 2\mu_X\mu_Y\\
	&=& \left(\mu_Y - \mu_X\right)^2 + \left(\sigma_Y - \sigma_X\right)^2 + 2\sigma_X\sigma_Y\left(1 - \corr{X}{Y}\right) \approx 1.722
\end{utregning}

\oppgave
\deloppgave
Siden $X \sim \mathrm{bin}(n, p)$ så er $\expected{X} = np$. Dette gir at
\begin{likning}
	\expected{\hat{p}} = \expected{\frac{X}{n}} = \frac{1}{n}\expected{X} = p
\end{likning}

\deloppgave
For en binomisk distribusjon så har vi at $\var{X} = np(1-p)$. Dette gir at
\begin{likning}
	\var{\hat{p}} = \frac{1}{n^2}\var{X} = \frac{p}{n}(1-p)
\end{likning}

\deloppgave
For at en estimator skal kunne bli klassifisert som en god estimator så krever vi at estimatoren er forventningsrett. Siden $\expected{\hat{p}} = p$ så er dette kravet oppfylt. Videre ønsker vi at variansen skal være så lite som mulig. Dette kravet handler mer om sammenligningen mellom to foreslåtte estimatorer og er ikke noe vi kan undersøke her. Tilslutt ønsker vi at variansen til estimatoren skal gå mot null når $n\to\infty$. Per utregning vist i forrige oppgave så ser vi at dette er tilfellet. Den oppgitte $\hat{p}$ er derfor en god estimator for $p$.

\deloppgave
Gitt $X = 354$ så har vi at $\hat{p} = 354/500$. En $95\%$ ($\alpha = 0.05$) konfidensinterval for $p$ er derfor:
\begin{utregning}
	\confidence{\hat{p}}{z}{\sqrt{\frac{\hat{p}}{n}(1-\hat{p})}} = \left[0.6681, 0.7479\right]
\end{utregning}

\deloppgave
Grunnen til at konfidensintervallet oppgitt ovenfor er gyldig er på grunn av sentral grenseteoremet. I henhold til sentral grenseteoremet er forutsetningene at $n\hat{p}(1-\hat{p}) > 5$. I vårt tilfellet så er $n\hat{p}(1-\hat{p}) = 103.368$ som er mer enn tilstrekkelig for å oppfylle det kravet.

\oppgave
For denne spesifikke problemstillingen velger vi å utføre en $T$-test fordi $\sigma$ er ukjent. La
\begin{utregning}
	H_0 \colon&& \mu \leq 15\\
	H_1 \colon&& \mu > 15
\end{utregning}
\begin{utregning}
	\bar{x} &=& \mean{x}{i}{1}{5} = 18.52
\end{utregning}
og
\begin{utregning}
	s^2 &=& \sVar{x}{i}{1}{5} = 18.177
\end{utregning}
Dette gir oss en
\begin{utregning}
	t &=& \tTest{\bar{x}}{15}{s}{5} = 1.846
\end{utregning}
Med en signifikansnivå $\alpha = 0.10$ så har vi
\begin{utregning}
	t_{\alpha}^{4} = 1.533
\end{utregning}
Siden $t > t_\alpha^4$ så forkaster vi $H_0$.

\oppgave
\deloppgave
Siden matrisen $A$ er en $2\times 2$-matrise så gjelder det at den karakteristiske likningen er
\begin{utregning}
	\det(A-\lambda I) &=& \lambda^2 - \tr{A}\lambda + \det(A)\\
	&=& \lambda^2 - 2\lambda - 2 = 0
\end{utregning}
Ved $abc$-formellen får vi at
\begin{likning}
	\lambda = \frac{2 \pm 2\sqrt{3}}{2} = 1\pm \sqrt{3}
\end{likning}
Egenverdiene til $A$ er derfor $\lambda_1 = 1-\sqrt{3}$ og $\lambda_2 = 1 + \sqrt{3}$.

For $\lambda_1$ så kan vi dedusere egenvektoren $\vec{v}_1$ til $A$:
\begin{utregning}
	\begin{pmatrix}
		\sqrt{3} & 3\\
		1 & \sqrt{3}
	\end{pmatrix}
	\sim
	\begin{pmatrix}
		1 & \sqrt{3}\\
		0 & 0
	\end{pmatrix}
\end{utregning}
som gir egenvektoren $\vec{v}_1 = (-\sqrt{3}, 1)$. Ved samme resonnement får vi $\vec{v}_2 = (\sqrt{3}, 1)$. $A$ kan derfor diagonaliseres:
\begin{likning}
	A = \frac{\sqrt{3}}{6}
	\begin{pmatrix}
	-\sqrt{3} & \sqrt{3}\\
	1 & 1
	\end{pmatrix}
	\begin{pmatrix}
		1 - \sqrt{3} & 0\\
		0 & 1 + \sqrt{3}
	\end{pmatrix}
	\begin{pmatrix}
		-1 & \sqrt{3}\\
		1 & \sqrt{3}
	\end{pmatrix}
\end{likning}

\deloppgave
La $D$ og $P$ være henholdsvis
\begin{utregning}
	D &=& \begin{pmatrix}
	1 - \sqrt{3} & 0\\
	0 & 1 + \sqrt{3}
\end{pmatrix}\\
	P &=& \begin{pmatrix}
		-\sqrt{3} & \sqrt{3}\\
		1 & 1
	\end{pmatrix}
\end{utregning}
Da har vi at
\begin{utregning}
	B &=& P\left(D^2 + 3D + 2I\right)P\inverse
\end{utregning}
Vi ser fra uttrykket ovenfor at egenverdiene til $B$ er
\begin{utregning}
	\lambda_1 &=& (1-\sqrt{3})^2 + 3(1-\sqrt{3}) + 2 = 9-5\sqrt{3}\\
	\lambda_2 &=& (1+\sqrt{3})^2 + 3(1+\sqrt{3}) + 2 = 9+5\sqrt{3}
\end{utregning}
Dette gir diagonalisering av $B$:
\begin{likning}
	B = \frac{\sqrt{3}}{6}
	\begin{pmatrix}
		-\sqrt{3} & \sqrt{3}\\
		1 & 1
	\end{pmatrix}
	\begin{pmatrix}
		9-5\sqrt{3} & 0\\
		0 & 9+5\sqrt{3}
	\end{pmatrix}
	\begin{pmatrix}
		-1 & \sqrt{3}\\
		1 & \sqrt{3}
	\end{pmatrix}
\end{likning}

\deloppgave
Observer at vi kan uttrykke systemet som $\vec{x}' = A\vec{x}$. Dette gir den generelle løsningen
\begin{likning}
	\vec{x} = c_1\vec{v}_1e^{\lambda_1 t} + c_2\vec{v}_2e^{\lambda_2 t}
\end{likning}
der $\lambda$ og $\vec{v}$ er henholdsvis egenverdiene og egenvektorene til $A$. Vi har derfor
\begin{likning}
	\vec{x} = c_1\begin{pmatrix}-\sqrt{3}\\1\end{pmatrix} e^{(1-\sqrt{3})t} + c_2\begin{pmatrix} \sqrt{3}\\1\end{pmatrix} e^{(1+\sqrt{3})t}
\end{likning}
Gitt initialbetingelsen $\vec{x}(0) = (0, 1)$ så har vi at
\begin{likning}
	\begin{pmatrix}
		-\sqrt{3} & \sqrt{3}\\
		1 & 1
	\end{pmatrix}\begin{pmatrix}
	c_1\\c_2
\end{pmatrix}
= \begin{pmatrix}
	0\\1
\end{pmatrix}
\end{likning}
Løser vi systemet får vi
\begin{likning}
	\begin{pmatrix}
		c_1\\c_2
	\end{pmatrix}
	= \frac{\sqrt{3}}{6}\begin{pmatrix}
		-1 & \sqrt{3}\\
		1 & \sqrt{3}
	\end{pmatrix}
	\begin{pmatrix}
		0\\1
	\end{pmatrix}
	= \frac{\sqrt{3}}{6}
	\begin{pmatrix}
		\sqrt{3}\\
		\sqrt{3}
	\end{pmatrix}
\end{likning}
Dette gir oss den partikulære løsningen:
\begin{likning}
	\vec{x}_p = \frac{1}{2}\begin{pmatrix}-\sqrt{3}\\1\end{pmatrix} e^{(1-\sqrt{3})t} + \frac{1}{2}\begin{pmatrix} \sqrt{3}\\1\end{pmatrix} e^{(1+\sqrt{3})t}
\end{likning}

\oppgave
\deloppgave
Vi kan uttrykke den oppgitte rekken som:
\begin{utregning}
	\sum_{n = 0}^\infty \frac{4^n - 3^n}{5^n} &=& \sum_{n = 0}^\infty \left(\frac{4}{5}\right)^n - \sum_{n = 0}^\infty \left(\frac{3}{5}\right)^n
\end{utregning}
Fra geometriske rekker vet vi at disse rekkene konvergerer til
\begin{utregning}
	\sum_{n = 0}^\infty \frac{4^n - 3^n}{5^n} &=& \frac{1}{1-4/5} - \frac{1}{1-3/5} = \frac{5}{2}
\end{utregning}

\deloppgave
\begin{enumerate}
	\item Ved divergenstesten så har vi at
	\begin{likning}
		\lim_{n \to \infty} \frac{2}{2+\left(8/9\right)^n} = 1
	\end{likning}
	så konkluderer vi med at rekken vil divergere.
	
	\item Vi bruker sammenligningstesten: La $b_n = n^{3/2}$ og $a_n$ være den oppgitte følgen. Da har vi at $0 \leq a_n \leq b_n$. Dersom $\sum b_n$ konvergerer så konvergerer $\sum a_n$. Vi kjenner igjen at $b_n$ er en $p$-rekke med $p < 1$ som tilsier at $b_n$ konvergerer. Dette vil si at den oppgitte rekken vil konvergere.
\end{enumerate}

\deloppgave
La $u = t^2$. Da har vi at
\begin{utregning}
	\sin u &=& \sum_{n = 0}^\infty \left(-1\right)^n\frac{u^{2n+1}}{\left(2n+1\right)!}\\
	\sin t^2 &=& \sum_{n = 0}^\infty \left(-1\right)^n\frac{t^{2(2n+1)}}{\left(2n+1\right)!}\\
	&=& \sum_{n = 0}^\infty \left(-1\right)^n\frac{t^{4n+2}}{\left(2n+1\right)!}\\
	&=& \sum_{n = 0}^\infty \left(-1\right)^n\frac{t^{4n}t^2}{\left(2n+1\right)!}
\end{utregning}
Dette gir at
\begin{utregning}
	\frac{\sin t^2}{t} &=& \sum_{n = 0}^\infty \left(-1\right)^n\frac{t^{4n}t^2}{t\left(2n+1\right)!}\\
	&=& \sum_{n = 0}^\infty \left(-1\right)^n\frac{t^{4n+1}}{\left(2n+1\right)!}
\end{utregning}
Dette vil si at
\begin{utregning}
	\int_0^x \frac{\sin t^2}{t} \dd t &=& \sum_{n = 0}^\infty \frac{\left(-1\right)^n}{\left(2n+1\right)!}\int_0^xt^{4n+1}\dd t\\
	&=& \sum_{n = 0}^\infty \frac{\left(-1\right)^n}{\left(2n+1\right)!}\left[\frac{t^{4n+2}}{4n+2}\right]_0^x\\
	&=& \sum_{n = 0}^\infty \frac{\left(-1\right)^n}{\left(2n+1\right)!}\cdot\frac{x^{2(2n+1)}}{2(2n+1)}
\end{utregning}

\oppgave
Tangentplanet til en funksjonen $g(x)$ er gitt ved $\indre{\nabla g(x, y, z)}{\vec{x} - \vec{x}_0} = 0$ der $\vec{x} = (x, y, z)$ og i dette tilfellet $\vec{x}_0 = (1, 1, 5)$.
\begin{utregning}
	\nabla g(x, y, z) = \begin{pmatrix}
		4x\\4y\\2z
	\end{pmatrix}
\end{utregning}
Dette gir
\begin{utregning}
	\indre{\nabla g(x, y, z)}{\vec{x} - \vec{x}_0} &=& \pdiff{g}{x}(x - x_0) + \pdiff{g}{y}(y-y_0) + \pdiff{g}{z}(z - z_0)\\
	&=& 4(x - 1) + 4(y - 1) + 10(z - 5)\\
	&=& 4x + 4y + 10z -58 \implies 2x + 2y + 5z = 29
\end{utregning}

\clearpage