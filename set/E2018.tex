\eksamen{2018}
\oppgave
Når $T\sim\mathrm{exp(\mu)}$ så er sannsynlighetstettheten gitt ved $f(t) = e^{-t/\mu}/\mu$. Det vil si at generelt
\begin{utregning}
	P(a < T < b) = \frac{1}{\mu}\int_{a}^b e^{-t/\mu}\dd t = e^{-a/\mu} - e^{-b/\mu}
\end{utregning}
Slik at
\deloppgave
\begin{likning}
	P(T > 90) = P(90 < T < \infty) = e^{-6/5} \approx 0.3012
\end{likning}
\deloppgave
\begin{likning}
	P(50 < T < 90) = e^{-2/3} - e^{-6/5} \approx 0.2122
\end{likning}
\deloppgave
\begin{likning}
	P(T\geq 150 | T > 90) = \frac{P(T \geq 150 \cap T > 90)}{P(T > 90)} = \frac{P(T\geq 150)}{P(T > 90)} = \frac{e^{-2}}{e^{-6/5}} \approx 0.4493
\end{likning}

\deloppgave
Vi har allerede fått oppgitt forventningen til populasjonen $\mu = 75$. Vi har da at
\begin{likning}
	\bar{T} = \frac{1}{20}\sum_{n = 1}^{20} T_n
\end{likning}
Og derfor
\begin{likning}
	E(\bar{T}) = \frac{1}{20}E\left(\sum_{n = 1}^{20} T_n\right)
\end{likning}
Siden $T_n$ er uavhengige og derfor
\begin{likning}
	E(\bar{T}) = \frac{1}{20} \sum_{n = 1}^{20} E(T_n)
\end{likning}
Siden hver av de målingene $T_n \sim \mathrm{exp}(\mu)$ så har vi at $E(T_n) = \mu$ for alle $n \in \mathbb Z_{21}\setminus{0}$. Dette gir
\begin{likning}
	E(\bar{T}) = \frac{1}{20}\cdot 20\mu = \mu
\end{likning}

\deloppgave
Per definisjon av $\sd^2(\bar{T})$ så har vi
\begin{utregning}
	\sd^2(\bar{T}) &=& \sd^2\left(\frac{1}{20}\sum_{n=1}^{20} T_n\right)\\
	&=& \frac{1}{400}\sd^2\left(\sum_{n=1}^{20} T_n\right)\\
	&=& \frac{1}{400}\sd^2\sum_{n=1}^{20}\sd^2\left(\bar{T}\right)\\
	&=& \frac{1}{400}\sd^2\sum_{n=1}^{20}\sigma^2\\
	&=& \frac{1}{400}\cdot 20\mu^2 = \frac{\mu^2}{20}
\end{utregning}
Dette gir $\sd(\bar{T}) = \sqrt{\sd^2(\bar{T})} = 16.77$.

\deloppgave
Siden $n = 20$ er betraktlig stor nok kan vi utnytte sentral grenseteoremet og anta at $\bar{T}\sim N\left(E(\bar{T}), \sd(\bar{T})\right)$. Med dette, kan vi derfor finne $P(\bar{T} > 90) = 1 - P(\bar{T}\leq 90)$, hvor
\begin{likning}
	P(\bar{T}\leq 90) = P\left(\frac{\bar{T}-\mu}{\sigma/\sqrt{n}} \leq \frac{90-\mu}{\sigma/\sqrt{n}}\right) = \Phi\left(Z \leq 0.8944\right) = 0.8133 
\end{likning}
Dette gir $P\left(\bar{T} > 90\right) = 0.1867$.

\deloppgave
Dersom $\bar{T} \sim \mathrm{exp}(\mu)$ så må $E(\bar{T}) = \sd(\bar{T})$. Siden vi fant ut at den ikke var det (i.e. $\bar{T}$ er ikke eksponentialfordelt) så kan vi desverre ikke regne ut en eksakt verdi for $\prob{\bar{T} > 90}$ ved å bruke eksponentialfordelingen.

\deloppgave
Null-hypotesen formuleres først ut ifra mistanken til problemstillingen som har blitt presentert. Siden mistanken her er at $\mu > 60$ så blir dette formuleringen for $H_1$. Det naturlige blir da å formulere $H_0$ som en komplement av $H_1$ altså $H_0 = \mu \leq 60$.

\deloppgave
Her utfører vi en $T$-test fordi $\sigma$ er ukjent. Gitt $\bar{t} = 75.3$ og $s = 50.7$ for $n = 30$ målinger så har vi
\begin{likning}
	T = \frac{\bar{t} - \mu_0}{s/\sqrt{n}} = 1.653
\end{likning}
Vi har også fra tabellen at $T_\alpha^{n-1} = 1.311$. Siden $T > T_\alpha^{n-1}$ så må ingeniørene konkludere med en $\alpha = 0.1$ signifikant-nivå at det tilstrekkelig med grunnlag å forkaste $H_0$.

\deloppgave
For en $90\%$ konfidensintervall av en $T$-test er vi ute etter å finne $T_{\alpha/2}^{n-1} = 1.699$. Vi har derfor at konfidensintervallen til $\mu$ er
\begin{likning}
	\left[\bar{t} - T_{\alpha/2}^{n-1}\frac{s}{\sqrt{n}}, \bar{t} + T_{\alpha/2}^{n-1}\frac{s}{\sqrt{n}}\right] = [59.57, 91.03]
\end{likning}

\oppgave
\deloppgave
Egenvierdiene $\lambda$ til en matrise $A$ er gitt ved å løse den karakteristiske likningen $\det(A-\lambda I) = 0$. Siden $A$ er en $2\times 2$-matrise kan vi benytte oss av at $\lambda_1 + \lambda_2 = \tr{A}$ og at $\lambda_1\lambda_2 = \det(A)$. Dette gir oss følgende:
\begin{utregning}
	\lambda_1 + \lambda_2 &=& 0\\
	\lambda_1\lambda_2 &=& -1
\end{utregning}
Løser vi likningen oppgitt ovenfor får vi at $\lambda_1 = \pm 1$. Dette medfølger at $\lambda_2 = \mp 1$. 
\newline

La $\lambda_1 = -1$ og $\lambda_2 = 1$. Egenvektorene $\vec{v}_1$ og $\vec{v}_2$ til $A$ får vi av å løse likningen:
\begin{likning}
	(A-\lambda I)\vec{v} = \vec{0}
\end{likning}
Med $\lambda_1 = -1$ får vi at
\begin{utregning}
	\begin{pmatrix}
		1 & 1\\
		1 & 1
	\end{pmatrix}&\sim&
	\begin{pmatrix}
		1 & 1\\
		0 & 0
	\end{pmatrix}
\end{utregning}
Dette gir oss $\vec{v}_1 = (1, -1)$. Ved samme argumentasjon for $\lambda_2 = 1$:
\begin{utregning}
	\begin{pmatrix}
		-1 & 1\\
		1 & -1
	\end{pmatrix} \sim
	\begin{pmatrix}
		1 & -1\\
		0 & 0
	\end{pmatrix}
\end{utregning}
får vi at $\vec{v}_2 = (1, 1)$.

Med egenverdiene og egenvektorene funnet over kan vi danne matrisen $P$ og matrisen $D$ slik at vi kan uttrykke $A$ som $A = PDP\inverse$. Hvor
\begin{likning}
	P = \begin{pmatrix}
		1 & 1\\
		-1 & 1
	\end{pmatrix} \implies P\inverse = \frac{1}{2}\begin{pmatrix}
	1 & -1\\
	1 & 1
\end{pmatrix}
\end{likning}
og 
\begin{likning}
	D = \begin{pmatrix}
		-1 & 0\\
		0 & 1
	\end{pmatrix}
\end{likning}
Dette gir oss
\begin{likning}
	A = \frac{1}{2}
	\begin{pmatrix}
		1 & 1\\
		-1 & 1
	\end{pmatrix}
	\begin{pmatrix}
		-1 & 0\\
		0 & 1
	\end{pmatrix}
	\begin{pmatrix}
		1 & -1\\
		1 & 1
	\end{pmatrix}
\end{likning}

Vi kan nå diagonalisere $B = A^7 + A^5 + I$ som følger:
\begin{likning}
	B = P\left(D^7 + D^5 + I\right)P\inverse
\end{likning}
egenverdiene til $B$ blir her $\lambda_1 = (-1)^7 + (-1)^5 + 1 = -1$ og $\lambda_2 = 3$. Dette gir diagonaliseringen av $B$:
\begin{likning}
	B = 
	\frac{1}{2}
	\begin{pmatrix}
		1 & 1\\
		-1 & 1
	\end{pmatrix}
	\begin{pmatrix}
		-1 & 0\\
		0 & 3
	\end{pmatrix}
	\begin{pmatrix}
		1 & -1\\
		1 & 1
	\end{pmatrix}
\end{likning}

\deloppgave
Omskriver vi systemet får vi 
\begin{utregning}
	x' &=& 0x + 1y\\
	y' &=& 1x + 0y
\end{utregning}
Her ser vi helt klart at vi kan uttrykke likningssystemet som
\begin{likning}
	\vec{x}' = A\vec{x}
\end{likning}
der $A$ er matrisen oppgitt i oppgaven.

Vi identifiserer at vi har et tilfelle av en homogent likning. For å løse dette må vi finne egenverdiene $\lambda$ og egenvektorene $\vec{v}$ til $A$. Dette har vi gjort tidligere. Systemet vil derfor ha en generell løsning gitt ved
\begin{likning}
	\vec{x} = c_1\vec{x}_1 + c_2\vec{x}_2
\end{likning}
hvor $c_1$ og $c_2$ er skalarer og 
\begin{utregning}
	\vec{x}_1 &=& \vec{v}_1 e^{\lambda_1 t}\\
	\vec{x}_2 &=& \vec{v}_2 e^{\lambda_2 t}
\end{utregning}
Med egenverdiene for $\lambda_1$ og $\lambda_2$ og henholdsvis egenvektorene $\vec{v}_1$ og $\vec{v}_2$ som vi tidligere fant for matrise $A$ har vi derfor at løsningen for systemet av differensiallikninger er gitt ved
\begin{utregning}
	\vec{x} &=& c_1\begin{pmatrix}1\\-1\end{pmatrix}e^{-t} + c_2 \begin{pmatrix}
		1 \\ 1
	\end{pmatrix} e^t
\end{utregning}
Ved initialbetingelsen $\vec{x}(0) = (1 ,1)$ har vi derfor at
\begin{utregning}
	\begin{pmatrix}
		1 & 1\\
		-1 & 1
	\end{pmatrix}\begin{pmatrix}
	c_1\\
	c_2
\end{pmatrix} &=& \begin{pmatrix}
	1\\1
\end{pmatrix}\\
\begin{pmatrix}
	c_1\\c_2
\end{pmatrix}
&=& \frac{1}{2}\begin{pmatrix}
	1 & -1\\
	1 & 1
\end{pmatrix}\begin{pmatrix}
	1\\1
\end{pmatrix} = \begin{pmatrix}
0\\1
\end{pmatrix}
\end{utregning}
Den partikulære løsningen er derfor
\begin{likning}
	\vec{x}_p = \begin{pmatrix}
		1\\1
	\end{pmatrix} e^t
\end{likning}

\oppgave
\deloppgave
\begin{enumerate}
	\item Rekken er en geometrisk rekke med felles forholdet $1/e < 1$. Dette tyder på at rekken vil konvergere mot:
	\begin{likning}
		\sum_{n = 1}^{\infty} \left(\frac{1}{e}\right)^n = \frac{1}{1-1/e} - 1 = \frac{1}{e-1}
	\end{likning}
	
	\item Her kan vi benytte oss av forholdstesten: La $a_n = (-3)^n/n$ da har vi at 
	\begin{utregning}
		\frac{a_{n+1}}{a_n} &=& \frac{(-3)^{n+1}}{n+1}\cdot \frac{n}{(-3)^n} = -\frac{3n}{n+1}
	\end{utregning}
	Vi får herifra at $\lim_{n\to\infty} \left|-3n/(n+1)\right| = 3 > 1$. Vi konkluderer derfor at rekken divergerer.
\end{enumerate}

\deloppgave
Vi vet at Maclaurin rekken til $e^{t}$ er gitt ved:
\begin{likning}
	e^t = \sum_{n = 0}^\infty \frac{t^n}{n!}
\end{likning}
Dette gir oss at
\begin{utregning}
	f(x) = 2xe^{x^2} &=& 2x\sum_{n = 0}^\infty \frac{x^{2n}}{n!}\\
	&=& \sum_{n = 0}^\infty \frac{2}{n!}x^{2n+1}
\end{utregning}
Konvergensradien til følgende rekke er 
\begin{utregning}
	R = \lim_{n \to \infty} \left| \frac{2}{n!} \cdot \frac{(n+1)!}{2} \right| = \lim_{n \to \infty} \left| n+1 \right| \to \infty
\end{utregning}

\oppgave
\deloppgave
Tangentplanet er gitt ved $\indre{\nabla f}{\Delta \vec{x}}$. La $f(x, y, z) = x^2 + y^3 - z^2 = 0$. Da har vi at
\begin{utregning}
	\indre{\nabla f}{\Delta \vec{x}} &=& 2x_0(x - x_0) + 3y_0^2(y - y_0) - 2z_0(z - z_0)\\
	&=& 2(x - 1) + 3(y - 1) - 4(z - 2)\\
	&=& 2x + 3y - 4z = -3
\end{utregning}

\deloppgave
Retningsvektoren måler hvor mye $g$ øker når vi beveger oss i retningen $\vec{u}$. Retningsvektoren er gitt ved:
\begin{likning}
	D_{\vec{u}} = \indre{\nabla g}{\frac{\vec{u}}{\norm{\vec{u}}}}
\end{likning}
Spesifikt for indreproduktet av et vektorrom så har du også at
\begin{likning}
	D_{\vec{u}} = \norm{\nabla g}\norm{\frac{\vec{u}}{\norm{\vec{u}}}}\cos\theta = \norm{\nabla g}\cos\theta
\end{likning}
Siden $\nabla g$ alltid peker mot der $g$ vokser fortest, la $\nabla g\parallel\vec{u}$. Da vil $\norm{\nabla g}$ være den største verdien av den retningsderiverte. Videre har vi også at
\begin{utregning}
	\norm{\nabla g}^2 = \indre{\nabla g}{\nabla g}
\end{utregning}
Med
\begin{utregning}
	\pdiff{g}{x} &=& -2xe^{-(x^2 + y^2)}\\
	\pdiff{g}{y} &=& -2ye^{-(x^2 + y^2)}
\end{utregning}
så har vi
\begin{likning}
	\norm{\nabla g}^2 = 4x^2e^{-2(x^2+y^2)} + 4y^2e^{-2(x^2 + y^2)} = 4(x^2 + y^2)e^{-2(x^2+y^2)}
\end{likning}
La $\theta = x^2 + y^2$ da har vi
\begin{likning}
	\norm{\nabla g}^2 = 4\theta e^{-2\theta}
\end{likning}
som gir
\begin{likning}
	\frac{\dd }{\dd \theta} 4\theta e^{-2\theta} = 4\left(1 - 2\theta^2\right)e^{-2\theta}
\end{likning}
Vi har at $\theta^2 = 1/2$ er et kritisk punkt for funksjonen over. Andre derivasjonstesten
\begin{likning}
	\frac{\dd^2}{\dd\theta^2} 4\theta e^{-2\theta} = 16(\theta - 1)e^{-2\theta}
\end{likning}
er negativ for $\theta = 1/\sqrt{2}$ som tyder på at $\theta^2 = 1/2$ er et topp punkt for $4\theta e^{-2\theta}$. Det vil si at alle punktene $(x, y)$ som $x^2 + y^2 = 1/2$ danner en sirkel om origo med radius $\sqrt{2}/2$ gir størst mulig verdi for den retningsderiverte av $g(x, y)$.

\clearpage